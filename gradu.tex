\nonstopmode
\batchmode %jos et halua nähdä
\documentclass[12pt,a4paper]{article}

\usepackage[head=0.98in,foot=0.98in]{geometry}

\title{Pro Gradu -tutkielma}

\usepackage[T2A]{fontenc}
\usepackage[utf8]{inputenc}
\usepackage[russian,greek,english,finnish]{babel}

\usepackage{indentfirst}
\usepackage{color}
\usepackage[nottoc,numbib]{tocbibind}

\usepackage{fancyhdr}
\pagestyle{myheadings}

%\usepackage{nanbib}	
%\bibpunct{(}{)}{;}{a}{,}{,}

\usepackage[autostyle=true, style=russian]{csquotes}
%saadaan venäläiset aakkoset listoihin
\usepackage{enumitem}
\makeatletter
\AddEnumerateCounter{\asbuk}{\russian@alph}{щ}
\makeatother

\usepackage[backend=biber,
        citestyle=authoryear-ibid,
        bibstyle=authortitle,
        sorting=nyt,
        dashed=false,
        maxnames=2,
        minnames=1,
        language=auto,%autobib laittaa et.al merkinnät kaikki venäjäksi
        autolang=other*,
        isbn=false,
        url=false,
        doi=false,
        %pagetracker=true,
        %citecounter=true,
        %citetracker=true,
        %ibidtracker=true,
        autocite=inline
        ]
{biblatex}

\addbibresource{viitteet.bib}

\setlength{\parindent}{4ex}
\setlength{\parskip}{0ex}% Kappaleiden välin asetus

\linespread{1.3}

\addto\captionsrussian{% Replace "english" with the language you use
  \renewcommand{\contentsname}%
    {Оглавнение}%
}

\addto\captionsrussian{% Replace "english" with the language you use
  \renewcommand{\bibname}
    {Список использоваемой литературы}
}

\addto\captionsrussian{% Replace "english" with the language you use
  \renewcommand{\refname}
    {Список использоваемой литературы}
}




\setlength{\voffset}{-0.02in}
\setlength{\hoffset}{0.57in}
\setlength{\marginparsep}{0pt}

% \usepackage{graphicx}
% tai \usepackage{graphics}
% kuvatiedostojen liittämiseksi

\begin{document}

\begin{titlepage}
\noindent
\begin{center}	
  \setlength{\parindent}{0mm}
  \sloppy
  \large \textsc{Хельсинкский университет}
  \vspace{5mm}

  \huge \textbf{Декадентский символизм <<Мелкого беса>>}
  \vspace{2mm}
  \textcolor{blue}\hrule
    \vspace{2mm}
    \large Dekadentskij simvolizm ''Melkogo besa''
    \linebreak \vfill   

  
%\vspace{20mm}
	
  \end{center}
  \vspace{15mm}
	\vfill
	
    \begin{flushright}
    	Tapio M. Pitkäranta \\
		Pro Gradu \\
		Venäjän kieli ja kirjallisuus \\
		Nykykielten laitos \\
		Helsingin yliopisto \\
		\today
	\end{flushright}
\end{titlepage}

%Valitaan tämän osuuden kieli
\selectlanguage{russian}

\tableofcontents
\thispagestyle{empty}

\pagebreak

%%% Oma teksti tämän jälkeen.
%%% Tyhjä rivi kappalten väliin.§] omiin tarkoituksiinsa.

\section{Введение}

Данная работа посвящена изучению отношений между романа Федора Сологуба <<Мелкий бес>> и сологубской теории искусства, которой он развивается в разных статьях, например,  <<Не постыдно ли быть декадентом>> (1896), <<Демоны поэтов>> (1907) <<Театр одной воли>> (1908) и <<Искусство наших дней>> (1915). Особое внимание уделяется сологубским мифом о Дульцинее и Альфонсе, который основан на романе <<Дон Кихот>>. В этом мифе Сологуб соединяет быт с искусством, грех со святостью и декадентство с символизмом. Многие исследователи уже заметили параллели  романа Сологуба с <<Дон Кихотом>>, а в данной работе изучаем подробно как эта тематика влияет на нарративе романа.

\section{Федор Сологуб в контексте русского модернизма конца XIX -- начала XX веков}

Творчество Федора Сологуба развивалось в рамках декадентства, русского модернистского литературного направления конца XIX века.
Самый значительный из сборников Сологуба – «Пламенный круг».
Главнимы мотивами в поэзии Сологуба являютя смерть,  

Пессимизм, неверие в возможность изменить социальную жизнь
беспомощность, малых возможностях человек
индивидуалиста рубежа веков
 не просто осознает, но и всячески культивирует свою отчужденность от общества.
tässä näkyy tiukka modernismin vaikutus. ETSI JOKU LÄHDE
«недотыкомка серая» (1899)
неотвязное наваждение, рожденное суеверием, диким, косным бытом, житейской пошлостью и отчаянием.   Tästä sanoo Blok jotain??

Частая тема поэзии Сологуба — власть дьявола над человеком («Когда я в бурном море плавал», 1902; «Чертовы качели», 1907). (431)
дьявол у Сологуба, как и у Бодлера, символизирует не только зло, царящее в мире, но выражает и бунтарский протест против обывательского благополучия и успокоенности.

 Поэтический словарь Сологуба пестрит словами смерть, труп, гроб, прах, склеп, могила, похороны, тьма, мгла.
 «Ибо все и во всем — Я, и только Я, и нет ничего, и не было и не будет», — таким утверждением в духе солипсизма завершается предисловие поэта к этому сборнику. Это же предисловие открыло первый том «Собрания сочинений» Сологуба в издательстве «Сирин».

\section{Сологубовская теория искусства}

В. В. Ерофеев в своей статье \citetitle{jerofeev1985} tulkitsee romaania «Мелкий бес». Hänen pääteesinään artikkelissa on, että F. Sologubin romaani on siirtymäkauden tuote, joka suhtautuu parodioivasti aiempaan venäläiseen romaanitraditioon, erityisesti kriittiseen realismiin, mutta myös muun muassa Gogolin, Puškinin ja Dostojevskin tuotantoon. Perinteisessä venäläisessä romaanissa on hänen mukaansa pohjalla käsitys totuuden, hyvyyden ja kauneuden keskinäisestä vastaavuudesta, sekä huolimatta mahdollisesta kriittisestä asenteesta yhteiskunnan epäkohtiin, kaiken alla oleva toivo mahdollisesta paremmasta huomisesta. Tämä toivo on Sologubin romaanissa kuitenkin vaihdettu kaipaukseen, joka ei voi toteutua. 

\enquote{калокагатия} \foreignlanguage{greek}{καλοκαγαθία}



Jerofejevin tulkinta romaanista, vaikka onkin paljon suppeampi, ei ole vastakkainen Z.G. Mintsin kanssa, joka artikkelissaan
З.Г. Минц в своей статье \citetitle{mints2004} laskee F. Sologubin romaanin osaksi uusmytologista romaanitraditiota, traditiota, joka huipentuu Andrei Belyin romaanissa «Петербург». Mytopoeettisen tekstin pohjalla on Vladimir Solovjovin ajattelu ja sen kolmijakoinen maailmankuva: Alun kaunis ja hyvä maailma, jonka kaaos tuhoaa ja josta pyritään kohti synteesiä. Uusmytologinen teksti ottaa tietoisesti vaikutteita eri lähteistä ja traditioista ja käyttää niitä hyväkseen rakentaakseen oman maailmakuvansa, oman myyttinsä.  Tyypillisesti yhtenä neomytologisen myytin lähteenä on oman aikansa arkitodellisuus. Sologubilla se on omaelämänkerrallisista aineistoista koottu, itse koettu maailma. Mints määrittelee, että Peredonovin myytti koostuu seuraavista aineksista:


\begin{enumerate}[itemsep=0mm, label=\asbuk*)]
\item realismin myytit provinssin elämästä Гоголь \enquote{Человек в футляре} Чехова.
\item \enquote{Мертвые души}
\item \enquote{Бесы}, \enquote{Идиот} Достоевского, безумие, хаос
\item \enquote{Пиковая дама} Пушкина, безумые, мотивы игровых карт
\item Недотыкомка, безумия нормальной мещанской жизни
\item Мифы, которые Передонов творит сам
\end{enumerate}

Näiden lisäksi eri tutkijat ovat lisänneet listaan esimerksi Don Quixoten \parencite{bagno2009} ja erilaiset raamatulliset motiivit ja myytit \parencite{kobrinski2013}.

Tässä siis yhtenä teksti-myyttinä on todellakin myös realistisen romaanin traditio, jota Jerofejev ansiokkaasti kuvailee tulkinnassaan, mutta se on vain osa suurempaa, ylemmän tason myyttiä, jossa ne vaikuttavat Mintsin kuvailemassa neomytologisen kirjallisuuden systeemissä. Näin ollen tässä systeemissä yksikertainen parodia ei ole mahdollinen tulkinta. Se edellyttäisi sitä, että romaanissa on kuuluvissa joku tekijän yhtenäinen ääni, joka parodioi venäläistä romaanitraditiota. Sen sijaan kyseessä on yksi lukuisista dikotomioista, teesi ja antiteesi, jonka osina on toisaalta romantiikan ironinen tyyli ja toisaalta realismin tyyli. 

Vastaavana dikotomiana voidaan nähdä myös dekadenssi ja symbolismi. Mints kirjoittaa tästä... Mutta kuitenkin oli kiistoja.... Näiden synteesinä voidaan nähdä uuden ajan modernistinen romaani, josta Sologub unelmoi jo vuonna 1896 kirjoittamassan julkaisemattomassa artikkelissaan \enquote{Не постыдно ли быть декадентом} \parencite{ref:sologub2007}. Siinä hän kirjoittaa:

\begin{quote}
Уже и теперь мы видим литературные произведения, в которых символизм облекался в формы декадентские, и которые производят глубокое впечатление. Будущее же в литературе принадлежит тому гению, который не убоится уничижительной клички декадента и с побеждающей художественной силой сочетает символическое мировоззрение с декадентскими формами.
\parencite[501.]{ref:sologub2007}
\end{quote}

По мнению Ерофеева, З.Г. Минц излишне мифологизирует романа, и он полагате, что мифологический материал в сюжетной линии с Людмилой и Сашей не связан напрамую с древней мифологией а с современной французской и русской литературной модой. Однако, Ерофеев 

Jerofejevin mielestä Mints mytologisoi romaania liikaa ja oikea lähde romaanin mytologisille aineksille on ajan muoti käyttää kirjallisuuden aineksena mytologisia aiheita. Ljudmilan unien tematiikka ei hänen mielestään ole peräisin suoraan antiikin myyteistä, vaan mutkan kautta ranskalaisesta kirjallisuudesta. \parencite[152.]{jerofeev1985} Jerofeejev ei kuitenkaan ota huomioon sitä, että Mintsin kuvaileman neomytologisen romaanin systeemissä myös nykyaikaisen ranskalaisen kirjallisuuden teksti-myytti voi tulla osaksi romaanin kuvailemaa «картина мир» mytologista tekstiä. Kuten Mints toteaa, myyttien lähtökohtana voi aivan yhtä hyvin olla antiikki kuin nykyajan todelliset tapahtumat. Jerofejevin huomi siis ei kumoa Mintsin artikkelin teesejä, vaan jopa vahvistaa niitä, koska hän tuo systeemiin vielä yhden mahdollisen myytin, joka lisää edelleen uusmytologiselle romaanilla ominaista dikotomiaa eri myyttien välillä. При этом, Mints itse huomauttaa, että ei aio tarkastella venäläisen symbolismin suhdetta venäläiseen. 



Jerofejevin tekemä huomio monikasvoisesta \enquote{многолик} kertojasta on kuitenkin asiaankuuluva, ja tätä aspektia ei Mintsin artikkelissa oteta huomioon. Uusmytologisen tekstin myyttien keskinäinen dikotomia ei voi olla näkymättä myös romaanin kerronnassa ja narrotologisessa rakenteessa. 

\section{Мелкий бес}

В этом главе рассматрываем основный и самый популярный роман Федора Сологуба <<Мелкий бес>> (закончен в 1902, впервые опубликован в 1905) с точки зрения нарратологий. Разные исследователи характеризировали сюжет романа либо как антисюжет -- очень симплистическая истрория в котором почти ничего не произходит, либо как максимально  
''Внешнего действия в «Мелком бесе» мало. Постепенно сходит с ума его главный герой, учитель гимназии Передонов, умственно ограниченный, угрюмый и недоброжелательный человек.''
\autocite[432.]{grigorjev1983}


\subsection{Ironia}

\enquote{В повестях и романах все глупости пишут.}\parencite[54]{sologub2004}

Tässä yritän todistaa romaanin ylimmän tason heterodiegeettisen kertojan ironisen suhtaumisen henkilöihinsä.
Tarkastellaan kahta valaisevaa tekstikatkelmaa.

\begin{quote}
Дома ждала Передонова важная новость. Еще в передней можно
было догадаться, что случилось необычное, – в горницах слышалась 
возня, испуганные восклицания. Передонов подумал, – не все
готово к обеду: увидели – он идет, испугались, торопятся. Ему стало
приятно, – как его боятся! Но оказалось, что произошло другое. 
Варвара выбежала в прихожую и закричала:

– Кота вернули!

\parencite[171]{sologub2004}.

\end{quote}

Katkelma sisältä kolmea eri tyyppistä kerrontaa. Ensimmäisessä virkkeessä
heterodiegeettinen kertoja ilmoittaa ennalta-aavistaen tärkeistä uutisista.
Kertojaääni ei voi tässä olla fokalisoituneena Peredonoviin, koska
hän ei voi tietää etukäteen tulevista uutisista. Tämän jälkeen tapahtumia
fokalisoidaan vapaalla epäsuoralla kerronnalla Peredonovin kautta.
%(vai onko suoraa kerrontaa, koska siinä on kohta передонов подумал. Persoonapronominit ja diskurssin vapaus viittaavat siihen, että perustellusti voidaan määritellä tämän olevan KHD.(Tammi, 42).
Hän kuulee ääniä ja tekee niistä omalaatuisen päätelmänsä, jonka mukaan hälinä
johtuu siitä, että häntä pelätään niin paljon: «– как его боятся!».
Vapaan epäsuoran kerronan paljastaa ensinnäkin viittaus aistihavaintoihin. Kerronnassa tulee ilmi Peredonovin kuulemat asiat. Väliin kommentoi taas kertoja: «Но оказалось, что произошло другое.» Sitten saadaan
Varvaran suoran esityksen kertomana kuulla mikä oli tämä "tärkeä" uutinen ja
paljastuu kertojan ironisoiva suhtautuminen tapahtumiin ja henkilöihin.
%\autocite[vrt.][230.]{hutchings1997}

Kissan palaaminen tuskin on objektiivisesti katsottuna erityisen tärkeä uutinen.
Silti Peredonovin sisääntulo kuvataan ikäänkuin kyse olisi kauhutarinan 
kohtauksesta. Heterodiegeettinen kertoja on siis oppositiossa kuvaamansa henkilön kanssa.

Toisessa kohtauksessa henkilöinä on Ljudmila ja koulun johtaja Hripatš. Kohtauksessa johtaja yrittää selvittää kumpi puhuu totta suhteesta Saša-poikaan, Ljudmila vai Peredonov.

\begin{quote}
Главное ее побуждение было,
конечно, сочувствие к мальчику, оскорбленному таким грубым 
подозрением, — желание заменить Саше отсутствующую семью, — и, 
наконец, он и сам такой славный, веселый и простодушный мальчик. 
Людмила даже заплакала, и быстрые маленькие слезинки 
удивительно-красиво покатились по ее розовым щекам, на ее смущенно-улыбающиеся
губы. \parencite[239]{sologub2004}
\end{quote}

Tässä huomaamme, kuinka heterodiegeettinen kertoja kuvaa Ljudmilan kertomusta ulkopuolisesta asemasta ironisoiden. Ironia näkyy täytesanojen,  kuten «конечно» käytössä, sekä ilmauksessa \enquote{слезинки 
удивительно-красиво покатились}. Kertoja ilmaisee kyynisen epäuskonsa Ljudmilan valheelliseen kertomukseen, mutta tälle kontrastina seuraavaksi fokalisaatio siirtyy koulun johtajaan:

\begin{quote}

— и нежным ароматом повеяло на Хрипача. И Хрипачу
вдруг захотелось сказать, что она «прелестна, как ангел небесный», и
что весь этот прискорбный инцидент «не стоит одного мгновения ее 
печали дорогой». Но он воздержался.

И журчал, и журчал нежный, быстрый Людмилочкин лепет, и 
развеивал дымом химерическое здание передоновской лжи. Только 
сравнить — безумный, грубый, грязный Передонов — и веселая, светлая,
нарядная, благоуханная Людмилочка. Говорит ли совершенную 
Людмила правду или привирает — это Хрипачу было все равно, — но он 
чувствовал, что не поверить Людмилочке, заспорить с нею, допустить 
какие-нибудь последствия, хоть бы взыскания с Пыльникова, — значило
бы попасться впросак и осрамиться на весь учебный округ. Тем более
что это связано с делом Передонова, которого, конечно, признают 
ненормальным. \parencite[240]{sologub2004}
\end{quote}

Jälleen fokalisaation muutos näkyy vapaana epäsuorana kerrontana, ja yhtenä merkkinä on taas aistihavainto: miellyttävä tuoksu. %toisena merkkinä täytesanat, конечно,
Tämän lisäksi Hripatšin vertaa ihanaa Ljudmilaa vielä uusmytologiselle tekstille tyypillisesti intertekstuaalisella viittauksella kahteen eri Lermontovin runoon, \enquote{Демон} ja \enquote{Тамара}. Тämä aiheuttaa vastakkainasettelun kerronnan sisällä ja lukijalle tulee tunne monikasvoisesta kertojasta. Tämän huomaa myös В.В. Ерофеев, kun tоteaa kertojan olevan \enquote{многолик, причем его лики с трудом совмещаются или не совмещаются вовсе} \parencite[157]{jerofeev1985}. Toinen vastakkainasettelu katkelman sisällä on Ljudmilan ja Peredonovin välillä eli totuuden ja valheen välillä. Ylemmän tason kertojan näkökulmasta on selvää että Ljudmila valehtelee, mutta Hripatšin näkökulmasta kaunis, Lermontovin ja romanttisen naiskuvan myytistä nouseva Ljudmila ei voi olla valheellinen, varsinkaan kun häntä vertaa rumaan, likaiseen, «epänormaaliin» Peredonoviin. Tässä tulee ehkä myös metatason viittaus koko romaanin sisältöön. Voiko hyvän ja kauniin sekä totuuden välille vetää suoraan yhtäläisyysmerkin? Onko antiikin \enquote{калокагатия} (\foreignlanguage{greek}{καλοκαγαθία}) riittävän kattava periaate romaanin tulkinnassa?


\section{Заключение}

\nocite{*}


\printbibheading[heading=bibliography,title={Список использованной литературы}]
\printbibliography[heading=subbibliography,keyword={primary},title={Источники}]
\printbibliography[heading=subbibliography,notkeyword={primary},title={Исследовательская литература}]


\end{document}
