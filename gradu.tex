\nonstopmode
% \batchmode jos et halua nähdä
\documentclass[12pt,a4paper]{article}

\title{Pro Gradu -tutkielma}

\usepackage[T2A]{fontenc}
\usepackage[utf8]{inputenc}
\usepackage[russian,finnish]{babel}

\usepackage{indentfirst}
\usepackage{color}
\usepackage[nottoc,numbib]{tocbibind}

%\usepackage{nanbib}	
%\bibpunct{(}{)}{;}{a}{,}{,}


\usepackage{csquotes}
\usepackage[backend=biber,
        citestyle=authoryear-ibid,
        bibstyle=authoryear,
        sorting=nyt,
        sortlocale=ru_RU,
        maxnames=2,
        minnames=1,
        language=auto,%autobib laittaa et.al merkinnät kaikki venäjäksi
        autolang=other*,
        isbn=false,
        url=false,
        doi=false,
        %pagetracker=true,
        %citecounter=true,
        %citetracker=true,
        %ibidtracker=true,
        autocite=plain
        ]
{biblatex}

\addbibresource{references.bib}




\setlength{\parindent}{4ex}
\setlength{\parskip}{0ex}% Kappaleiden välin asetus

\linespread{1.3}

\addto\captionsrussian{% Replace "english" with the language you use
  \renewcommand{\contentsname}%
    {Оглавнение}%
}

\addto\captionsrussian{% Replace "english" with the language you use
  \renewcommand{\bibname}
    {Список использоваемой литературы}
}

\addto\captionsrussian{% Replace "english" with the language you use
  \renewcommand{\refname}
    {Список использоваемой литературы}
}

\setlength{\voffset}{-0.02in}
\setlength{\hoffset}{0.57in}
\setlength{\marginparsep}{0pt}

% \usepackage{graphicx}
% tai \usepackage{graphics}
% kuvatiedostojen liittämiseksi

\begin{document}

\begin{titlepage}
\noindent
\begin{center}	
  \setlength{\parindent}{0mm}
  \sloppy
  \large \textsc{Хельсинкский университет}
  \vspace{5mm}

  \huge \textbf{Моя работа}
  \vspace{2mm}
  \textcolor{blue}\hrule
    \vspace{2mm}
    \large Pro Gradu -tutkielma
    \linebreak \vfill   

  
%\vspace{20mm}
	
  \end{center}
  \vspace{15mm}
	\vfill
	
    \begin{flushright}
    	Tapio M. Pitkäranta \\
		Pro Gradu \\
		Venäjän kieli ja kirjallisuus \\
		Nykykielten laitos \\
		Helsingin yliopisto \\
		\today
	\end{flushright}
\end{titlepage}

%Valitaan tämän osuuden kieli
\selectlanguage{russian}

\tableofcontents

\pagebreak

%%% Oma teksti tämän jälkeen.
%%% Tyhjä rivi kappalten väliin.§] omiin tarkoituksiinsa.

\section{Федор Сологуб в контексте русского модернизма конца XIX -- начала XX веков}

Творчество Федора Сологуба развивалось в рамках декадентства, русского модернистского литературного направления конца XIX века.
Самый значительный из сборников Сологуба – «Пламенный круг».
Главнимы мотивами в поэзии Сологуба являютя смерть,  

Пессимизм, неверие в возможность изменить социальную жизнь
беспомощность, малых возможностях человек
индивидуалиста рубежа веков
 не просто осознает, но и всячески культивирует свою отчужденность от общества.
tässä näkyy tiukka modernismin vaikutus. ETSI JOKU LÄHDE
«недотыкомка серая» (1899)
неотвязное наваждение, рожденное суеверием, диким, косным бытом, житейской пошлостью и отчаянием.   Tästä sanoo Blok jotain??

Частая тема поэзии Сологуба — власть дьявола над человеком («Когда я в бурном море плавал», 1902; «Чертовы качели», 1907). (431)
дьявол у Сологуба, как и у Бодлера, символизирует не только зло, царящее в мире, но выражает и бунтарский протест против обывательского благополучия и успокоенности.

 Поэтический словарь Сологуба пестрит словами смерть, труп, гроб, прах, склеп, могила, похороны, тьма, мгла.
 «Ибо все и во всем — Я, и только Я, и нет ничего, и не было и не будет», — таким утверждением в духе солипсизма завершается предисловие поэта к этому сборнику. Это же предисловие открыло первый том «Собрания сочинений» Сологуба в издательстве «Сирин».




\section{Мелкий бес}

В этом главе рассматрываем основный и самый популярный роман Федора Сологуба <<Мелкий бес>> (закончен в 1902, впервые опубликован в 1905) с точки зрения нарратологий. Разные исследователи характеризировали сюжет романа либо как антисюжет -- очень симплистическая истрория в котором почти ничего не произходит, либо как максимально  
''Внешнего действия в «Мелком бесе» мало. Постепенно сходит с ума его главный герой, учитель гимназии Передонов, умственно ограниченный, угрюмый и недоброжелательный человек.''
\parencite[432]{grigorjev1983}.

\section{Введение}

OОбыкновенный текст, который должен быть читаемым. И ещё немножечко. Краткая ссылка: \textcite[150--152]{kobrinski2013} отметит что–то. Обычные ''кавычки'', <<русские кавычки>> и «Unicode». \parencite[200]{kobrinski2013}.

Toisessa kappaleessa kerrotaan lisää. \parencite{shapir2007}. Täydellinen viittaus teokseen,
jolla on useampi tekijä näyttää vielä täydellisempänä
tältä \parencite[][268]{SKS2007}, mutta kun viitataan heti uudellen, tulee tällainen tulos
\parencite[50]{SKS2007}. Jos useamman tekijän lähde on venäläinen, viittaus näyttää tältä
\parencite{ljustrova1976}. Tässä viitataan teokseen, jolle on määritelty kääntäjä
\parencite[25]{ref:sologub1918}.

\subsection{Ironia}

<<В повестях и романах все глупости пишут.>>\parencite[54]{sologub2004}



\begin{quote}
Дома ждала Передонова важная новость. Еще в передней можно
было догадаться, что случилось необычное, – в горницах слышалась 
возня, испуганные восклицания. Передонов подумал, – не все
готово к обеду: увидели – он идет, испугались, торопятся. Ему стало
приятно, – как его боятся! Но оказалось, что произошло другое. 
Варвара выбежала в прихожую и закричала:

– Кота вернули!
<<<<<<< HEAD
\parencite[171]{sologub2004}.
=======

\parencite[200]{ref:sologub2001}.
>>>>>>> 21db4c977b7da7ce719f35d52604a7d9f1ce8d72
\end{quote}

Katkelma sisältä kolmea eri tyyppistä kerrontaa. Ensimmäisessä virkkeessä
heterodiegeettinen kertoja ilmoittaa ennalta-aavistaen tärkeistä uutisista.
Kertojaääni ei voi tässä olla fokalisoituneena Peredonoviin, koska
hän ei voi tietää etukäteen tulevista uutisista. Tämän jälkeen tapahtumia
fokalisoidaan vapaalla epäsuoralla kerronnalla Peredonovin kautta. Hän
kuulee ääniä ja tekee niistä omalaatuisen päätelmänsä, jonka mukaan hälinä
johtuu siitä, että häntä pelätään niin paljon: «– как его боятся!». Väliin 
kommentoi kertoja: «Но оказалось, что произошло другое.» Sitten saadaan
Varvaran suoran esityksen kertomana kuulla mikä oli tämä "tärkeä" uutinen ja
paljastuu kertojan ironisoiva suhtautuminen tapahtumiin ja henkilöihin. \parencite[vrt.][230]{hutchings1997}.

Kissan palaaminen tuskin on objektiivisesti katsottuna erityisen tärkeä uutinen.
Silti Peredonovin sisääntulo kuvataan ikäänkuin kyse olisi kauhutarinan 
kohtauksesta.

\subsection{Что-то ещё}

Huomaa, että lähde ei tule lähdeluetteloon, ellei siihen jotenkin
viitata \parencite[55]{artjunova1983}. Kokeillaan testata jotain.


\printbibliography[heading=bibintoc,title={Список использоваемой литературы}]

\end{document}
