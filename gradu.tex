\nonstopmode
\batchmode %jos et halua nähdä
\documentclass[12pt,a4paper]{article}

\title{Pro Gradu -tutkielma}

\usepackage[T2A]{fontenc}
\usepackage[utf8]{inputenc}
\usepackage[russian,english,finnish]{babel}

\usepackage{indentfirst}
\usepackage{color}
\usepackage[nottoc,numbib]{tocbibind}


%\usepackage{nanbib}	
%\bibpunct{(}{)}{;}{a}{,}{,}


\usepackage{csquotes}
\usepackage[backend=biber,
        citestyle=authoryear-ibid,
        bibstyle=authortitle,
        sorting=nyt,
        dashed=false,
        maxnames=2,
        minnames=1,
        language=auto,%autobib laittaa et.al merkinnät kaikki venäjäksi
        autolang=other*,
        isbn=false,
        url=false,
        doi=false,
        %pagetracker=true,
        %citecounter=true,
        %citetracker=true,
        %ibidtracker=true,
        autocite=inline
        ]
{biblatex}

\addbibresource{viitteet.bib}




\setlength{\parindent}{4ex}
\setlength{\parskip}{0ex}% Kappaleiden välin asetus

\linespread{1.3}

\addto\captionsrussian{% Replace "english" with the language you use
  \renewcommand{\contentsname}%
    {Оглавнение}%
}

\addto\captionsrussian{% Replace "english" with the language you use
  \renewcommand{\bibname}
    {Список использоваемой литературы}
}

\addto\captionsrussian{% Replace "english" with the language you use
  \renewcommand{\refname}
    {Список использоваемой литературы}
}




\setlength{\voffset}{-0.02in}
\setlength{\hoffset}{0.57in}
\setlength{\marginparsep}{0pt}

% \usepackage{graphicx}
% tai \usepackage{graphics}
% kuvatiedostojen liittämiseksi

\begin{document}

\begin{titlepage}
\noindent
\begin{center}	
  \setlength{\parindent}{0mm}
  \sloppy
  \large \textsc{Хельсинкский университет}
  \vspace{5mm}

  \huge \textbf{Декадентский символизм <<Мелкого беса>>}
  \vspace{2mm}
  \textcolor{blue}\hrule
    \vspace{2mm}
    \large Dekadentskij simvolizm ''Melkogo besa''
    \linebreak \vfill   

  
%\vspace{20mm}
	
  \end{center}
  \vspace{15mm}
	\vfill
	
    \begin{flushright}
    	Tapio M. Pitkäranta \\
		Pro Gradu \\
		Venäjän kieli ja kirjallisuus \\
		Nykykielten laitos \\
		Helsingin yliopisto \\
		\today
	\end{flushright}
\end{titlepage}

%Valitaan tämän osuuden kieli
\selectlanguage{russian}

\tableofcontents

\pagebreak

%%% Oma teksti tämän jälkeen.
%%% Tyhjä rivi kappalten väliin.§] omiin tarkoituksiinsa.

\section{Введение}

Данная работа посвящена изучению отношений между романа Федора Сологуба <<Мелкий бес>> и сологубской теории искусства, которой он развивается в разных статьях, например,  <<Не постыдно ли быть декадентом>> (1896), <<Демоны поэтов>> (1907) <<Театр одной воли>> (1908) и <<Искусство наших дней>> (1915). Особое внимание уделяется сологубским мифом о Дульцинее и Альфонсе, который основан на романе <<Дон Кихот>>. В этом мифе Сологуб соединяет быт с искусством, грех со святостью и декадентство с символизмом. Многие исследователи уже заметили параллели  романа Сологуба с <<Дон Кихотом>>, а в данной работе изучаем подробно как эта тематика влияет на нарративе романа.

\section{Федор Сологуб в контексте русского модернизма конца XIX -- начала XX веков}

Творчество Федора Сологуба развивалось в рамках декадентства, русского модернистского литературного направления конца XIX века.
Самый значительный из сборников Сологуба – «Пламенный круг».
Главнимы мотивами в поэзии Сологуба являютя смерть,  

Пессимизм, неверие в возможность изменить социальную жизнь
беспомощность, малых возможностях человек
индивидуалиста рубежа веков
 не просто осознает, но и всячески культивирует свою отчужденность от общества.
tässä näkyy tiukka modernismin vaikutus. ETSI JOKU LÄHDE
«недотыкомка серая» (1899)
неотвязное наваждение, рожденное суеверием, диким, косным бытом, житейской пошлостью и отчаянием.   Tästä sanoo Blok jotain??

Частая тема поэзии Сологуба — власть дьявола над человеком («Когда я в бурном море плавал», 1902; «Чертовы качели», 1907). (431)
дьявол у Сологуба, как и у Бодлера, символизирует не только зло, царящее в мире, но выражает и бунтарский протест против обывательского благополучия и успокоенности.

 Поэтический словарь Сологуба пестрит словами смерть, труп, гроб, прах, склеп, могила, похороны, тьма, мгла.
 «Ибо все и во всем — Я, и только Я, и нет ничего, и не было и не будет», — таким утверждением в духе солипсизма завершается предисловие поэта к этому сборнику. Это же предисловие открыло первый том «Собрания сочинений» Сологуба в издательстве «Сирин».

\section{Сологубская теория искусства}

\section{Мелкий бес}

В этом главе рассматрываем основный и самый популярный роман Федора Сологуба <<Мелкий бес>> (закончен в 1902, впервые опубликован в 1905) с точки зрения нарратологий. Разные исследователи характеризировали сюжет романа либо как антисюжет -- очень симплистическая истрория в котором почти ничего не произходит, либо как максимально  
''Внешнего действия в «Мелком бесе» мало. Постепенно сходит с ума его главный герой, учитель гимназии Передонов, умственно ограниченный, угрюмый и недоброжелательный человек.''
\autocite[432.]{grigorjev1983}


\subsection{Ironia}

Будущее же в литературе принадлежит тому гению, который не убоится уничижительной клички декадента и с побеждающей художественной силой сочетает символическое мировоззрение с декадентскими формами

<<В повестях и романах все глупости пишут.>>\parencite[54]{sologub2004}



\begin{quote}
Дома ждала Передонова важная новость. Еще в передней можно
было догадаться, что случилось необычное, – в горницах слышалась 
возня, испуганные восклицания. Передонов подумал, – не все
готово к обеду: увидели – он идет, испугались, торопятся. Ему стало
приятно, – как его боятся! Но оказалось, что произошло другое. 
Варвара выбежала в прихожую и закричала:

– Кота вернули!

\parencite[171]{sologub2004}.

\end{quote}

Katkelma sisältä kolmea eri tyyppistä kerrontaa. Ensimmäisessä virkkeessä
heterodiegeettinen kertoja ilmoittaa ennalta-aavistaen tärkeistä uutisista.
Kertojaääni ei voi tässä olla fokalisoituneena Peredonoviin, koska
hän ei voi tietää etukäteen tulevista uutisista. Tämän jälkeen tapahtumia
fokalisoidaan vapaalla epäsuoralla kerronnalla Peredonovin kautta. (vai onko suoraa
kerrontaa, koska siinä on kohta передонов подумал. Persoonapronominit ja diskurssin vapaus viittaavat siihen, että perustellusti voidaan määritellä tämän olevan KHD.
(Tammi, 42). Hän
kuulee ääniä ja tekee niistä omalaatuisen päätelmänsä, jonka mukaan hälinä
johtuu siitä, että häntä pelätään niin paljon: «– как его боятся!». его paljastaa
että kyseessä on KHD. Väliin
kommentoi kertoja: «Но оказалось, что произошло другое.» Sitten saadaan
Varvaran suoran esityksen kertomana kuulla mikä oli tämä "tärkeä" uutinen ja
paljastuu kertojan ironisoiva suhtautuminen tapahtumiin ja henkilöihin. \autocite[vrt.][230.]{hutchings1997}

Kissan palaaminen tuskin on objektiivisesti katsottuna erityisen tärkeä uutinen.
Silti Peredonovin sisääntulo kuvataan ikäänkuin kyse olisi kauhutarinan 
kohtauksesta.

\section{Заключение}

\nocite{*}


\printbibheading[heading=bibliography,title={Список использованной литературы}]
\printbibliography[heading=subbibliography,keyword={primary},title={Источники}]
\printbibliography[heading=subbibliography,notkeyword={primary},title={Исследовательская литература}]


\end{document}
