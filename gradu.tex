\nonstopmode
\batchmode %jos et halua nähdä
\documentclass[12pt,a4paper]{article}
\usepackage{verbatim}
\usepackage[head=0.98in,foot=0.98in]{geometry}

\title{Pro Gradu -tutkielma}

\usepackage[T2A]{fontenc}
\usepackage[utf8]{inputenc}
\usepackage[russian,greek,english,finnish]{babel}

\usepackage{indentfirst}
\usepackage{color}
\usepackage[nottoc,numbib]{tocbibind}

\usepackage{fancyhdr}
\pagestyle{myheadings}

%\usepackage{nanbib}	
%\bibpunct{(}{)}{;}{a}{,}{,}

\usepackage[autostyle=true, style=russian]{csquotes}
%saadaan venäläiset aakkoset listoihin
\usepackage{enumitem}
\makeatletter
\AddEnumerateCounter{\asbuk}{\russian@alph}{щ}
\makeatother

\usepackage[backend=biber,
        citestyle=authoryear-ibid,
        bibstyle=authortitle,
        sorting=nyt,
        dashed=false,
        maxnames=2,
        minnames=1,
        language=auto,%autobib laittaa et.al merkinnät kaikki venäjäksi
        autolang=other*,
        isbn=false,
        url=false,
        doi=false,
        %pagetracker=true,
        %citecounter=true,
        %citetracker=true,
        %ibidtracker=true,
        autocite=inline
        ]
{biblatex}

\addbibresource{viitteet.bib}

\setlength{\parindent}{4ex}
\setlength{\parskip}{0ex}% Kappaleiden välin asetus

\linespread{1.3}

\addto\captionsrussian{% Replace "english" with the language you use
  \renewcommand{\contentsname}%
    {Оглавнение}%
}

\addto\captionsrussian{% Replace "english" with the language you use
  \renewcommand{\bibname}
    {Список использоваемой литературы}
}

\addto\captionsrussian{% Replace "english" with the language you use
  \renewcommand{\refname}
    {Список использоваемой литературы}
}




\setlength{\voffset}{-0.02in}
\setlength{\hoffset}{0.57in}
\setlength{\marginparsep}{0pt}

% \usepackage{graphicx}
% tai \usepackage{graphics}
% kuvatiedostojen liittämiseksi

\begin{document}

\begin{titlepage}
\noindent
\begin{center}	
  \setlength{\parindent}{0mm}
  \sloppy
  \large \textsc{Хельсинкский университет}
  \vspace{5mm}

  \huge \textbf{Декадентский символизм <<Мелкого беса>>}

%TEOSTEN NIMET KURSIVOIDAAN
%-artikkelit, novellit, runot, lainausmerkkeihin

  \vspace{2mm}
  \textcolor{blue}\hrule
    \vspace{2mm}
    \large Dekadentskij simvolizm ''Melkogo besa''
    \linebreak \vfill   

  
%\vspace{20mm}
	
  \end{center}
  \vspace{15mm}
	\vfill
	
    \begin{flushright}
    	Tapio M. Pitkäranta \\
		Pro Gradu \\
		Venäjän kieli ja kirjallisuus \\
		Nykykielten laitos \\
		Helsingin yliopisto \\
		\today
	\end{flushright}
\end{titlepage}

%Valitaan tämän osuuden kieli
\selectlanguage{russian}

\tableofcontents
\thispagestyle{empty}

\pagebreak

%%% Oma teksti tämän jälkeen.
%%% Tyhjä rivi kappalten väliin.§] omiin tarkoituksiinsa.

\begin{comment}
\section{Введение}

Данная работа посвящена изучению отношений между романа Федора Сологуба <<Мелкий бес>> и сологубской теории искусства, которой он развивается в разных статьях, например,  <<Не постыдно ли быть декадентом>> (1896), <<Демоны поэтов>> (1907) <<Театр одной воли>> (1908) и <<Искусство наших дней>> (1915). Особое внимание уделяется сологубским мифом о Дульцинее и Альфонсе, который основан на романе <<Дон Кихот>>. В этом мифе Сологуб соединяет быт с искусством, грех со святостью и декадентство с символизмом. Многие исследователи уже заметили параллели  романа Сологуба с <<Дон Кихотом>>, а в данной работе изучаем подробно как эта тематика влияет на нарративе романа.

\section{Федор Сологуб в контексте русского модернизма конца XIX -- начала XX веков}

Творчество Федора Сологуба развивалось в рамках декадентства, русского модернистского литературного направления конца XIX века.
Самый значительный из сборников Сологуба – «Пламенный круг».
Главнимы мотивами в поэзии Сологуба являютя смерть,  

Пессимизм, неверие в возможность изменить социальную жизнь
беспомощность, малых возможностях человек
индивидуалиста рубежа веков
 не просто осознает, но и всячески культивирует свою отчужденность от общества.
tässä näkyy tiukka modernismin vaikutus. ETSI JOKU LÄHDE
«недотыкомка серая» (1899)
неотвязное наваждение, рожденное суеверием, диким, косным бытом, житейской пошлостью и отчаянием.   Tästä sanoo Blok jotain??

Частая тема поэзии Сологуба — власть дьявола над человеком («Когда я в бурном море плавал», 1902; «Чертовы качели», 1907). (431)
дьявол у Сологуба, как и у Бодлера, символизирует не только зло, царящее в мире, но выражает и бунтарский протест против обывательского благополучия и успокоенности.

 Поэтический словарь Сологуба пестрит словами смерть, труп, гроб, прах, склеп, могила, похороны, тьма, мгла.
 «Ибо все и во всем — Я, и только Я, и нет ничего, и не было и не будет», — таким утверждением в духе солипсизма завершается предисловие поэта к этому сборнику. Это же предисловие открыло первый том «Собрания сочинений» Сологуба в издательстве «Сирин».

 
\end{comment}

\section{Неомифологический \enquote{Мелкий бес}}

В. В. Ерофеев в своей статье \citetitle{jerofeev1985} (1985) анализирует роман «Мелкий бес». Его основным положением в статье является то, что «Мелкий бес» Ф. Сологуба -- роман переходного периода в пути к модернистской прозе, который относиться пародийно к русской романной традиции, особенно к критическому реализму, но также к творчеству Гоголя, Пушкина и Достоевского. \parencite[145.]{jerofeev1985}

Ерофеев полагает, что: \enquote{автор романа не приемлет той жизни, о которой повествует}. Повествователь в романе описывает идиотизм и садизм персонажей, пошлость жизни в маленьком городе и относится ко всему этому по большей части иронично. Образ автора, по мнению Ерофеева, образ прогрессивной фигуры, у которого прогрессизм -- \enquote{безнадежный, бесперспективный \enquote{прогрессизм}}. \parencite[146.]{jerofeev1985}

По Ерофееву, на основе традиционного русского романа является понятие о соответствии красоты, истины и добры, и несмотря на критическое отношение автора к социальной действительности, у автора все таки есть надежда что есть путь к лучшему обществу в будущем. Эта надежда в романе Ф. Сологуба превратилась в тоску, которая не может совершаться.  \parencite[158.]{jerofeev1985}


Истолкование Ерофеева, хотя оно не настолько обширное, не противоречит истолкованию З.Г. Минц, которая в  своей статье \citetitle{mints2004} \parencite*{mints2004} рассматривает вопрос о романе Сологуба как о части традиции \enquote{неомифологических} текстов, традиции, которая достигнет кульминации в романе Андрея Белого \enquote{Петербург}. \enquote{Неомифологический} роман по Минц -- в тесном отношении к предыдущей традиции: \enquote{Ориентация на архаическое сознание непременно соединяется в \enquote{неомифологических} текстах с проблематикой и структурой социального романа, повести и т. д., а зачастую — и с полемикой с ними.} \parencite[60.]{mints2004} 

На основе мифопоэтического текста является понятие \enquote{панэстетизма}. В \enquote{панэстетизме} сущность мира -- Красота. И для символистов на рубеже XIX -- XX веков -- самая высшая красота -- искусство. Искусство также является единственным средством через которое можно получить истинное знание о мире. Для символистов сам мир был мифом или символом и символистское искусство, творя мифы, при этом превращает мир. На картину мира символистов также влияла философия Владимира Соловьева и его идея о триадной сущности мира. По Соловьеву, мир находится в постоянном состоянии битвы, где Божественный Космос борется с Хаосом за Душу мира. Из этого следует миф о мире, который находится в состоянии изменения и в котором бесконечно следуют друг за другом \enquote{теза}(совершенное состояние, где существует только Божество), \enquote{антитеза}(мир разбивается Хаосом) и  \enquote{синтеза} (небесное и земное соединяются опять в всеединое). Из этого мифа о мире и следует диалектический характер \enquote{неомифологического} текста, где на вид противопоставляющие понятия существуют в одном тексте: \enquote{\enquote{Земное} хотя и ощущается как \enquote{отблеск} и \enquote{отзвук} небесного, как \enquote{антитеза} духовной \enquote{тезы} бытия, однако наделяется самостоятельностью, самоценностью и не меньшей, чем \enquote{теза}, ролью в становлении мирового универсума.} \parencite[71.]{mints2004}


\enquote{Неомифологический} текст принимает осознанно материал из разных источников и традиции и использует их чтобы построить свою картину мира, свои миф. Поэтому \enquote{неомифологический} текст -- в сущности интертекстуальный. Как показывает Минц, одним из типично важных источников \enquote{неомифологического} мифа является современная общественная действительность. Для Ф. Сологуба, это материал частично из его собственной жизни. Минц определяет, что в романе \enquote{Мелкий бес}, миф Передонова формируется из следующих источников:

\begin{enumerate}[itemsep=0mm, label=\asbuk*)]
\item Текст-миф реализма о жизни в периферии (Гоголь, \enquote{Человек в футляре} Чехова).
\item \enquote{Мертвые души}.
\item Безумие, хаос \enquote{Бесы}, \enquote{Идиот} Достоевского.
\item \enquote{Пиковая дама} Пушкина, безумие, мотивы игровых карт.
\item Его собственное стихотворение \enquote{Недотыкомка}, безумие обычной мещанской жизни
\item Мифы, которые Передонов творит сам, мания преследования.
\end{enumerate}

%Исследователи добавляли в списке также, в частности, \enquote{Дон Кихот} %Сервантеса \parencite{bagno2009} и разные библейские мотиви и мифи %\parencite{kobrinski2013}.

 
Как показывала Минц, одним из текст-мифов действительно является традиция реалистического русского романа, которую Ерофеев описывает в своей статье, но в \enquote{Мелком бесе} он просто частью более обширного мифа высшего уровня, вместе с другими, в описываемой Минц системе \enquote{неомифологического} текста.

По мнению Ерофеева, З.Г. Минц излишне мифологизирует роман, и он полагает, что мифологический материал в сюжетной линии с Людмилой и Сашей не связан напрямую с древней мифологией а с современной Сологубе французской и русской литературной модой \parencite[152]{jerofeev1985}.

М.М. Павлова довольно убедительно показала, что на раннее творчество Сологуба действительно влиял европейский натурализм и декаданс конца XIX века, особенно роман Ж.К. Гюсманса \enquote{Наоборот} (\enquote{À rebours}) Павлова пишет: \enquote{Он внес также существенные изменения в первоначальный замысел \enquote{Мелкого беса}, дополнив его \enquote{эротической} сюжетной линией: \enquote{парфюмерным} романом Людмилы и Саши.} \parencite*[168.]{pavlova2007} При этом Сологуб в своей неопубликованной статье 1896 года \citetitle{ref:sologub2007} обширно обсуждает французскую литературу.

Ерофеев, однако, не учитывает, что в системе \enquote{неомифологического} романа, который описывает Минц, текст-миф современной французской литературы может подобным образом стать частью мифологии и «картины мира» \enquote{неомифологического} текста. Таким образом, наблюдение Ерофеева о влиянии литературной моды не опровергает гипотезу Минц, а даже укрепляет ее, так как он добавит в систему еще один текст-миф, который вводит еще одну свойственную \enquote{неомифологическому} роману дихотомию между мифами.

Следовательно, если мы соглашаемся с Минц, что \enquote{Мелкий бес} -- \enquote{неомифологический} роман, то не можем согласиться с Ерофеевым с тем, что роман просто пародия, так как это означал бы, что в романе существует какая-то едииная авторская позиция или точка зрения. Как пишет Минц:

\begin{quote}
Поэтому именно отнесенность смысла символистского текста к универсальному «мифу о мире» оказывается единственной относительно авторитетной точкой зрения этого текста, наиболее слышным «голосом» в «полифоническом» произведении. \emph{Голос автора} в символистском «неомифологическом» произведении \emph{есть определение места изображаемого в универсальном космогоническом мифе.}
\parencite[77.]{mints2004}
\end{quote}

Вместо того, противопоставление между новым романом и традицией является лишь одной из нескольких дихотомии, тезой и антитезой, на основе которых в текст-мифе стремятся к синтезу в картине мира романа.

В роли такой же дихотомии может быть и декадентство и символизм. В 1890-е годы в России разгорелась полемика о формах нового искусства. Зинайда Гиппиус писала в статье, опубликованной в журнале \enquote{Северный Вестник} 1896 г., так: 

\begin{quote}
Символизм, прежде всего, диаметрально противоположен декадентству. Быть может, даже не стоило бы упоминать о декадентстве рядом с символизмом. [---] эти два понятия так печально смешались в умах людей даже наиболее почтенных, что невольно хочется разделить их навсегда. [---] Декадентство боялось смерти и умерло, больное и слабое. Декадентство боялось разума, чистоты понимания. Символизм весь в свете разума, в его широком и ясном спокойствии. \footnote{Северный вестник. № 12. 1896. с. 235-246. цит. по \cite[150]{pavlova2007}.}
\end{quote}

По мнению З.Г. Минц, в исторической перспективе, символизм и декадентство не противостояли так диаметрально, а на практике они были лишь разными \enquote{полюсами притяжения}, между которыми писатели двигались, принимая разные характеристики того или другого полюса.\parencite[62]{mints2004}. В романе Ф.Сологуба можно видеть попытку синтеза этих полюсов. Об этой попытке он написал в своей статье, таким образом:

\begin{quote}
Уже и теперь мы видим литературные произведения, в которых символизм облекался в формы декадентские, и которые производят глубокое впечатление. Будущее же в литературе принадлежит тому гению, который не убоится уничижительной клички декадента и с побеждающей художественной силой сочетает символическое мировоззрение с декадентскими формами.
\parencite[501.]{ref:sologub2007}
\end{quote}

Перцов keskustelee dekadenssin voitosta ja sen tulevaisuudesta. \parencite[514]{pavlova2016}

\section{Мелкий бес}

В этом главе рассматриваем основный и самый популярный роман Федора Сологуба <<Мелкий бес>> (закончен в 1902, впервые опубликован в 1905) с точки зрения нарратологий. Разные исследователи характеризовали сюжет романа либо как антисюжет -- очень симплистическая история в котором почти ничего не происходит, либо как максимально  
\enquote{Внешнего действия в «Мелком бесе» мало. Постепенно сходит с ума его главный герой, учитель гимназии Передонов, умственно ограниченный, угрюмый и недоброжелательный человек.}
\autocite[432.]{grigorjev1983}


\subsection{Терминология}

В нашем анализе, мы употребляем терминологию из монографии Вольфа Шмида \enquote{Нарратология} (2008), так как данная книга является самим ясным и систематичным введением по этой теме на русском языке, найдено нами. Потому-что терминология, употребляемая Шмидом, немного отличается от традиции русского литературоведения, в частности от терминологии Женетта, мы описываем значению некоторых терминов.
  
По мнение Шмида, термин \enquote{повествователь} -- проблематичен, так как он
определяется по-разному. При этом, и в советском и в современном
литературоведении термины \enquote{образ автора} и \enquote{образ повествователя} употребляли как
синонимы \parencite[67--68]{schmid2008}.  Как мы заметили выше, у В.В.
Ерофеева именно этот термин смешивается с термином «автор». Поэтому мы в нашем анализе
употребляем термин нарратор, потому-что он является как можно больше
нейтральным. Особенно в неомифологическом романе, как мы показывали высше, мы не можем определить какой-то единой авторской позиции.  Когда речь идет о разных уровнях нарратива, употребляем термины
первичный нарратор, вторичный нарратор и т.д. Первичным называется тот нарратор,
который  обрамляется истории и вторичным повествователя интекста или вставной истории.

В категоризации Шмида противопоставление \enquote{диегетический -- недиегетический} также немного отличается от терминологии Женетта. По словам Шмида: \enquote{... если \enquote{я} <нарратора T.P.> относится только к акту повествования, то нарратор является недиегетическим, если же \enquote{я} относится то к акту повествования, то к повествуемому миру — диегетическим.} \parencite[84]{schmid2008}.

\subsubsection{Fokalisaatio}
Kertojan diskurssi, henkilön diskurssi ja kertojan-henkilön diskurssi

\subsection{Образ нарратора}

Miksi haluamme analysoida kertojaa? 
Что мы знаем о нарраторе в романе \enquote{Мелкий бес}? Ja onko kertoja niin "monikasvoinen" kuin Jerofeev artikkelissaan esittää? Narratiivissa on muutamia implisiittisiä ja eksplisiittisiä merkkejä kertojasta. Kertoja vaikuttaa enimmäkseen kaikkitietävältä ja kaikkiallaolevalta, hän pystyy välittämään tietoa henkilöiden ajatuksista, motivaatioista ja tunteista. \enquote{Он не знал, о чем говорить с Мартою. Она была ему нелюбопытна,
как все предметы, с которыми не были кем-то установлены для него
приятные или неприятные отношения.
} (18). Mutta välillä kertojan tieto on ehdollisempaa: \enquote{Хоть он и узнал наверное, что смеялись не над ним, но в нем осталась досада, — так после прикосновения жгучей крапивы долго остается и  возрастает боль, хотя уже крапива и далече.} (19). Sanan \enquote{наверное} käyttö korostaa kertojan tiedon ehdollisuutta. Toisaalta siteerattu kohta on hiukan ambivalentti sen suhteen, onko kyseessä текст нарратора vai kertojan Peredonovin kautta fokalisoima diskurssi. Kertojan diskurssin puolesta puhuu se, että kasvit ja kasvuston tarkka kuvailu on toistuvana motiivina kertojalla ja tässäkin verrataan tunnetta nokkosen poltteeseen. Tämä ei sopisi yhtä hyvin Peredonoville. 

Kertoja käyttää enimmäkseen neutraalia kirjakieltä, päin vastoin kuin romaanin henkilöt, jotka viljelevät kielessään runsaasti murreilmaisuja, sananlaskuja ja sanaleikkejä. Согласно Джулиан Коннолли, henkilöiden puhe menettää koherenssiaan loppua kohden ja tämä kuvastaa temaattisesti kasvavaa kaaosta \parencite[358--359]{connolly1981}. TÄHÄN TOINENKIN VIITTAUS. Välillä, varsinkin esittäessään omia mielipiteitään, kertoja siirtyy ylempään kielen rekisteriin: \enquote{... – и воистину в нашем веке надлежит красоте быть попранной и поруганной.} (51).

Kertoja kuitenkin tuntee myös paikallisen puheenparren. Tästä saamme eksplisiittisen todisteen ensimmäisen kerran luvussa IV: \enquote{ – А вы ее зовите Клавдюшкой.
Варваре это понравилось. Она повторяла:
— Клавдюшка, дюшка.
И смеялась скрипучим смехом. Надо заметить, что дюшками в 
нашем городе называют свиней.} \parencite[33]{sologub2004}. Tämä merkittävä huomautus toistuu vielä toisen kerran samassa luvussa: \enquote{Поэтому она чаще ездила на извозчиках,
хотя больших расстояний в нашем городе не было.} \parencite[34]{sologub2004}. Nämä huomautukset paitsi osoittavat, että kertoja ymmärtää paikallista murretta, myös antavat ilmi, että kertoja on saman kaupungin asukas kuin missä kertomus tapahtuu.

Asenteellisesti kertoja asettuu useimmiten kuvaamiensa henkilöiden yläpuolelle. Hän arvostelee heitä eksplisiittisesti, esimerkiksi kohtauksessa, missä kuvataan Peredonovin ja Ljudmilan lemmenyötä: \enquote{И это восхитительное тело для этих двух пьяных и грязных людишек являлось только источником низкого соблазна.} (51). Implisiittisesti kertoja arvostelee henkilöitä esimerkiksi siten, että kerronnassa korostuvat henkilöiden ja tapahtumien nurjat puolet, kurjuus, idiotismi jne. 
Tässä korostuu личность нарратора, koska hänellä on tapahtumiin ja henkilöihin kohdistuvia ajatuksia ja mielipiteitä. Mutta välillä myös kertojan diskurssiin tarttuu henkilöiden sadistisia ominaisuuksia. Esimerkiksi otamme kuvauksen Грушина n ulkonäöstä, kun hänet esitellään lukijalle ensimmäisen kerran:

\begin{quote}
На беглый взгляд она не то
чтоб казалась очень грязною, а производила такое впечатление, словно
она никогда не моется, а только выколачивается вместе со своими 
платьями. Думалось, что если ударить по ней несколько раз камышовкою,
то поднимется до самого неба пыльный столб.
(33)
\end{quote}



Tässä tulemme jälleen kertojan monikasvoisuuden ongelmaan, minkä Jerofeev esitteli. Me esitämme, että kertojan moniulotteisuus kuvastaa nimenomaan uusmytologisen romaanin rakennetta ja on sen väistämätön tulos. Kertoja ei voisi olla muuta kuin monilähtöinen, kaksinapainen. Kertojan henkilössä kuvastuu toisaalta naturalistinen romaani. Hän kuvailee pikkukaupungin poshlostia, henkilöiden nurjia puolia, idiotismia. Mutta toisaalta samaan kertojaan yhdistyy myös symbolistinen puoli, juuri siihen tapaan kuin Sologub sanoi: символическое мировоззрение с декадентскими формами. Jos tarkastellaan aiemmin siteerattua Grushinan kuvausta, siinä romaanihenkilön kepillä hakkaamiseen yhdistyy tomun symboliikka. Kertoja ei tyydy sanomaan, että Grushina oli likainen, vaan hän käyttää keinona hyperbolaa: Grushinaa hakkaamalla syntyy taivaaseen asti nouseva pölypatsas. Согласно Т. Венцелова, у русских символистов, начиная с стихотворение В. Брюсова \emph{Демоны пыли} (1899)\footnote{Ioanna Brjusovan mukaan runoilija sai inspiraation pölyiseen huoneeseen tulleen valonsädettä katsoessaa. Myös tässä proosallisen arkinen pöly muuttui symboliseksi. В. Брюсов. Собр. соч. в семи томах. М., 1973. Т. 1. с. 600.}, пыл ассоциируется с демонами, хаосом, энтропией и сама Вершина в романе связывается с бесом женского пола \parencite[42--44]{venclova2012}. Samaa mieltä on myös Pustygina artikkelissaan \enquote{Символика огня в романе Федора Сологуба \enquote{Мелкий бес}}: \begin{quote}
Символика дыма, в частности, связывается с образом Вершиной — «черной колдуньи», постоянно курящей и развешивающей вокруг завесы дыма. Этот персонаж «Мелкого беса» двойствен: прежде всего в нем заключена символика смерти, а также и всей действительности (причем дым в последнем случае соотносится с «покрывалом майи»). Символом дыма как бы еще раз подчеркивается иллюзорность, «кажимость» вещного мира, одной из его многочисленных ложных оболочек. Одно «обличие мира», сгорев, оставляет после себя прах; круг, таким образом, замыкается.
\end{quote}

Myös ympäristön kuvauksissa esiintyy sama:
Странная, жестокая улыбка сквозь слезы озарила ее лицо, как 
ярко-пылающий луч на закате сквозь последнее падение усталого дождя.(142) valo pölyn läpi, 
тряпкою, лентою, веткою, флагом, тучкою, собачкою, столбом пыли на улице, и везде ползет и бежит за Передоновым, — измаяла, истомила его зыбкою своею пляскою. Хоть бы кто-нибудь избавил, словом каким (198)




\subsubsection{Недиегетический нарратор}

Kysymys

\subsubsection{Картина мира нарратора}

\subsection{Ympäristön kuvaus, henkilöt vs kertoja}

\subsection{Ironia}

\enquote{В повестях и романах все глупости пишут.} \parencite[54]{sologub2004}



Tässä yritän todistaa romaanin kertojan ironisen suhtaumisen henkilöihinsä, sekä vastakkainasettelun, joka usein näkyy fokalisoitujen henkilöiden ja ironisen kertojan välillä.
% * <ida.reini@gmail.com> 2018-03-09T11:25:46.960Z:
% 
% romaanin ylin taso kuulostaa jotenkin oudolta. enkö oo vaan tottunut kerronnan teoriaan vai pitäiskö laittaa ainakin 'romaanin kerronnan ylimmän tason'. tai ehkä jossain aiemmassa kappaleessa aukaiset lyhyesti näitä kerronnan tasoja.
% 
% ^ <tapio.m.pitkaranta@gmail.com> 2018-03-09T12:27:19.494Z:
% 
% On itselläkin vielä vähän hukassa. Täytyy kerrata, etsiä venäläiset vastineet ja varmasti avata alkuun jotenkin. Fokalisoimaton kertoja tms.
%
% ^.

Tarkastellaan kahta valaisevaa tekstikatkelmaa. Текст нарратора merkitään [ТН] Косвенная речь [КР]. Прямая речь [ПР]

\begin{quote}
Дома ждала Передонова важная новость. Еще в передней можно
было догадаться, что случилось необычное, – [НПР] в горницах слышалась 
возня, испуганные восклицания. Передонов подумал, – не все
готово к обеду: увидели – он идет, испугались, торопятся. Ему стало
приятно, – как его боятся! Но оказалось, что произошло другое. 
Варвара выбежала в прихожую и закричала:

– [ПР] Кота вернули!

\parencite[171.]{sologub2004}

\end{quote}

Katkelmassa kertojan ironinen asenne tuodaan ilmi fokalisaation avulla. Lainauksen ensimmäinen lause on kertojan näkökulmasta. 
Kertoja ei voi tässä olla fokalisoituneena Peredonoviin, koska
hän ei voi tietää etukäteen tulevista uutisista. Tämän jälkeen tapahtumia
fokalisoidaan epäsuoralla kerronnalla Peredonovin kautta.
%(vai onko suoraa kerrontaa, koska siinä on kohta передонов подумал. Persoonapronominit ja diskurssin vapaus viittaavat siihen, että perustellusti voidaan määritellä tämän olevan KHD.(Tammi, 42).
Hän kuulee ääniä ja tekee niistä omalaatuisen päätelmänsä, jonka mukaan hälinä
johtuu siitä, että häntä pelätään niin paljon: «– как его боятся!».
Peredonovin näkökulman paljastaa ensinnäkin viittaus aistihavaintoihin. Kerronnassa tulee ilmi Peredonovin kuulemat asiat. Väliin kommentoi taas kertoja: «Но оказалось, что произошло другое.» Sitten saadaan
Varvaran suoran esityksen kertomana kuulla mikä oli tämä \enquote{tärkeä} uutinen ja
paljastuu kertojan ironisoiva suhtautuminen tapahtumiin ja henkilöihin.
%\autocite[vrt.][230.]{hutchings1997}

Kissan palaaminen tuskin on objektiivisesti katsottuna erityisen tärkeä uutinen.
Silti Peredonovin sisääntulo kuvataan ikäänkuin kyse olisi kauhutarinan 
kohtauksesta. Перычный нарратор on siis oppositiossa kuvaamansa henkilön kanssa.
% * <ida.reini@gmail.com> 2018-03-09T11:36:10.168Z:
% 
% Tähän kaipais ehkä lopetukseksi jonkun yhteenvedon tyyliin
% "ironia luodaan siis tässä kohtauksessa liioittelun keinoin" tjsp muotoile sopivammaksi
% 
% ^.
Toisessa kohtauksessa henkilöinä on Ljudmila ja koulun johtaja Hripatš. Kohtauksessa johtaja yrittää selvittää kumpi puhuu totta suhteesta Saša-poikaan, Ljudmila vai Peredonov.

\begin{quote}
Главное ее побуждение было,
конечно, сочувствие к мальчику, оскорбленному таким грубым 
подозрением, — желание заменить Саше отсутствующую семью, — и, 
наконец, он и сам такой славный, веселый и простодушный мальчик. 
Людмила даже заплакала, и быстрые маленькие слезинки 
удивительно-красиво покатились по ее розовым щекам, на ее смущенно-улыбающиеся
губы. \parencite[239]{sologub2004}
\end{quote}

Tässä huomaamme, kuinka heterodiegeettinen kertoja kuvaa Ljudmilan kertomusta ulkopuolisesta asemasta ironisoiden. Ironia näkyy täytesanojen,  kuten «конечно» käytössä, sekä ilmauksessa \enquote{слезинки 
удивительно-красиво покатились}. Kertoja ilmaisee kyynisen epäuskonsa Ljudmilan valheelliseen kertomukseen, mutta tälle kontrastina seuraavaksi fokalisaatio siirtyy koulun johtajaan:
%mutta lukija tietää nyt jo mikä on varsinainen totuus, joten kuka voisi tulkita tämän väärin?
\begin{quote}

— и нежным ароматом повеяло на Хрипача. И Хрипачу
вдруг захотелось сказать, что она «прелестна, как ангел небесный», и
что весь этот прискорбный инцидент «не стоит одного мгновения ее 
печали дорогой». Но он воздержался.

И журчал, и журчал нежный, быстрый Людмилочкин лепет, и 
развеивал дымом химерическое здание передоновской лжи. Только 
сравнить — безумный, грубый, грязный Передонов — и веселая, светлая,
нарядная, благоуханная Людмилочка. Говорит ли совершенную 
Людмила правду или привирает — это Хрипачу было все равно, — но он 
чувствовал, что не поверить Людмилочке, заспорить с нею, допустить 
какие-нибудь последствия, хоть бы взыскания с Пыльникова, — значило
бы попасться впросак и осрамиться на весь учебный округ. Тем более
что это связано с делом Передонова, которого, конечно, признают 
ненормальным. \parencite[240]{sologub2004}
\end{quote}

Jälleen fokalisaation muutos näkyy epäsuorana kerrontana, ja yhtenä merkkinä on taas aistihavainto: miellyttävä tuoksu, jonka johtaja haistaa. %toisena merkkinä täytesanat, конечно,
Tämän lisäksi Hripatšin vertaa ihanaa Ljudmilaa vielä uusmytologiselle tekstille tyypillisesti intertekstuaalisella viittauksella kahteen eri Lermontovin runoon, \enquote{Демон} ja \enquote{Тамара}. Hripatš ei toki muista miten sitaatti jatkuu: \enquote{... Прекрасна, как ангел небесный, | Как демон, коварна и зла...} Toisaalta taas Peredonoviin viitataan mytologisoivalla ilmauksella \enquote{химерическое здание передоновской лжи}, joka viittaa usein dekadenttien runoudessa valheeseen ja petokseen nimen omaan naisen hahmossa, olihan myyttinen khimaira naispuolinen epäsikiö. \parencite[200]{hansen-love1999}. Myyttien kerroksellinen kasautuminen ja niiden välinen ristiriita Hripatšin kautta fokalisoituneessa kerronnassa verrattuna fokalisoimattoman kertojan kylmään ironiaan aiheuttaa vastakkainasettelun kerronnan sisällä ja lukijalle tulee tunne monikasvoisesta kertojasta. Tämän huomaa myös В.В. Ерофеев, kun tоteaa kertojan olevan \enquote{многолик, причем его лики с трудом совмещаются или не совмещаются вовсе} \parencite[157]{jerofeev1985}. Toinen vastakkainasettelu katkelman sisällä on Ljudmilan ja Peredonovin välillä eli totuuden ja valheen välillä. Ylemmän tason kertojan näkökulmasta on selvää että Ljudmila valehtelee, mutta Hripatšin näkökulmasta kaunis, Lermontovin, ja laajemmin romanttisen kirjallisuuden naiskuvan, myytistä nouseva Ljudmila ei voi olla valheellinen, varsinkaan kun häntä vertaa rumaan, likaiseen, «epänormaaliin» Peredonoviin. Ensimmäisen tason kertojan näkökulmasta on vain tyypillinen pikkukaupungin tapahtuma, toisin sanoen пошлость. Johtajan mielessä tapahtuvan sitaatin myötä se nousee kuitenkin myyttisiin sfääreihin ja samalla osapuolet sekoittuvat. Lermontovin Tamara on runossa itse asiassa kaunis, mutta petollinen ja houkuttelee miehiä vain tuhotakseen heidät. Sitaatin täydentävä toinen säe itse asiassa lähentää Hripatšin ja kertojan näkökantoja ja tuo teesin ja antiteesin kohti synteesiä. Tässä tulee ehkä myös metatason viittaus koko romaanin sisältöön. Voiko hyvän ja kauniin sekä totuuden välille vetää suoraan yhtäläisyysmerkin? Onko antiikin \enquote{калокагатия} (\foreignlanguage{greek}{καλοκαγαθία}) riittävän kattava periaate romaanin tulkinnassa?
% * <ida.reini@gmail.com> 2018-03-09T11:48:58.367Z:
% 
% oon myös vähän hukassa noiden tasojen kanssa, oonkohan ikinä operoinut noiden kanssa. onko ylempi taso siis sama kuin ensimmäinen taso? en tiedä pitääkö sun selventää vai mun sivistyä
% 
% ^.
% * <ida.reini@gmail.com> 2018-03-09T11:45:11.121Z:
% 
% "Tämä aiheuttaa vastakkainasettelun kerronnan sisällä ja lukijalle tulee tunne...."
% Hetkonen. Tässä kohtaa vähän tipuin kärryltä. Vastakkainasettelu on siis enkelissä ja demonissa vai miehessä ja naisessa? Eli avaatko mikä "tämä"
% 
% ^ <tapio.m.pitkaranta@gmail.com> 2018-03-09T12:35:30.107Z:
% 
%  Myyttien kerroksellinen kasautuminen ja niiden välinen ristiriita Hripatšin kautta fokalisoituneessa kerronnassa verrattuna fokalisoimattoman kertojan kylmään ironiaan aiheuttaa vastakkainasettelun kerronnan sisällä
%
% ^.
% * <ida.reini@gmail.com> 2018-03-09T11:43:16.868Z:
% 
% "olihan myyttien khimaira naispuolinen epäsikiö" on tämän viikon suosikkilauseeni :--------D
% 
% ^.


\section{Заключение}

%\nocite{*}


\printbibheading[heading=bibliography,title={Список использованной литературы}]
\printbibliography[heading=subbibliography,keyword={primary},title={Источники}]
\printbibliography[heading=subbibliography,notkeyword={primary},title={Исследовательская литература}]


\end{document}
