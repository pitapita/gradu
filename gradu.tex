\nonstopmode
% \batchmode jos et halua nähdä
\documentclass[12pt,a4paper]{article}

\title{Pro Gradu -tutkielma}

\usepackage[T2A]{fontenc}
\usepackage[utf8]{inputenc}
\usepackage[russian,finnish]{babel}

\usepackage{indentfirst}
\usepackage{color}
\usepackage[nottoc,numbib]{tocbibind}

%\usepackage{nanbib}	
%\bibpunct{(}{)}{;}{a}{,}{,}


\usepackage{csquotes}
\usepackage[backend=biber,
        citestyle=authoryear-ibid,
        bibstyle=authoryear,
        sorting=nyt,
        sortlocale=ru_RU,
        maxnames=2,
        minnames=1,
        language=auto,%autobib laittaa et.al merkinnät kaikki venäjäksi
        autolang=other*,
        isbn=false,
        url=false,
        doi=false,
        %pagetracker=true,
        %citecounter=true,
        %citetracker=true,
        %ibidtracker=true,
        autocite=plain
        ]
{biblatex}

\addbibresource{references.bib}




\setlength{\parindent}{4ex}
\setlength{\parskip}{0ex}% Kappaleiden välin asetus

\linespread{1.3}

\addto\captionsrussian{% Replace "english" with the language you use
  \renewcommand{\contentsname}%
    {Оглавнение}%
}

\addto\captionsrussian{% Replace "english" with the language you use
  \renewcommand{\bibname}
    {Список использоваемой литературы}
}

\addto\captionsrussian{% Replace "english" with the language you use
  \renewcommand{\refname}
    {Список использоваемой литературы}
}

\setlength{\voffset}{-0.02in}
\setlength{\hoffset}{0.57in}
\setlength{\marginparsep}{0pt}

% \usepackage{graphicx}
% tai \usepackage{graphics}
% kuvatiedostojen liittämiseksi

\begin{document}

\begin{titlepage}
\noindent
\begin{center}	
  \setlength{\parindent}{0mm}
  \sloppy
  \large \textsc{Хельсинкский университет}
  \vspace{5mm}

  \huge \textbf{Моя работа}
  \vspace{2mm}
  \textcolor{blue}\hrule
    \vspace{2mm}
    \large Pro Gradu -tutkielma
    \linebreak \vfill   

  
%\vspace{20mm}
	
  \end{center}
  \vspace{15mm}
	\vfill
	
    \begin{flushright}
    	Tapio M. Pitkäranta \\
		Pro Gradu \\
		Venäjän kieli ja kirjallisuus \\
		Nykykielten laitos \\
		Helsingin yliopisto \\
		\today
	\end{flushright}
\end{titlepage}

%Valitaan tämän osuuden kieli
\selectlanguage{russian}

\tableofcontents

\pagebreak

%%% Oma teksti tämän jälkeen.
%%% Tyhjä rivi kappalten väliin.§] omiin tarkoituksiinsa.

\section{Введение}

OОбыкновенный текст, который должен быть читаемым. И ещё немножечко. Краткая ссылка: \textcite[150--152]{kobrinski2013} отметит что–то. Обычные ''кавычки'', <<русские кавычки>> и «Unicode». \parencite[200]{kobrinski2013}.

Toisessa kappaleessa kerrotaan lisää. \parencite{shapir2007}. Täydellinen viittaus teokseen,
jolla on useampi tekijä näyttää vielä täydellisempänä
tältä \parencite[][268]{SKS2007}, mutta kun viitataan heti uudellen, tulee tällainen tulos
\parencite[50]{SKS2007}. Jos useamman tekijän lähde on venäläinen, viittaus näyttää tältä
\parencite{ljustrova1976}. Tässä viitataan teokseen, jolle on määritelty kääntäjä
\parencite[25]{ref:sologub1918}.

\subsection{Что-то ещё}

Huomaa, että lähde ei tule lähdeluetteloon, ellei siihen jotenkin
viitata \parencite[55]{artjunova1983}. Kokeillaan testata jotain.


\printbibliography[heading=bibintoc,title={Список использоваемой литературы}]

\end{document}
