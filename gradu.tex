\nonstopmode
\batchmode %jos et halua nähdä
\documentclass[12pt,a4paper]{article}
\usepackage{verbatim}
\usepackage[head=0.98in,foot=0.98in]{geometry}

\title{Pro Gradu -tutkielma}

\usepackage[T2A]{fontenc}
\usepackage[utf8]{inputenc}
\usepackage[russian,greek,english,finnish]{babel}

\usepackage{indentfirst}
\usepackage{color}
\usepackage[nottoc,numbib]{tocbibind}

\usepackage{fancyhdr}
\pagestyle{myheadings}

%\usepackage{nanbib}	
%\bibpunct{(}{)}{;}{a}{,}{,}

\usepackage[autostyle=true, style=russian]{csquotes}
%saadaan venäläiset aakkoset listoihin
\usepackage{enumitem}
\makeatletter
\AddEnumerateCounter{\asbuk}{\russian@alph}{щ}
\makeatother

\usepackage[backend=biber,
        citestyle=authoryear-ibid,
        bibstyle=authortitle,
        sorting=nyt,
        dashed=false,
        maxnames=2,
        minnames=1,
        language=auto,%autobib laittaa et.al merkinnät kaikki venäjäksi
        autolang=other*,
        isbn=false,
        url=false,
        doi=false,
        %pagetracker=true,
        %citecounter=true,
        %citetracker=true,
        %ibidtracker=true,
        autocite=inline
        ]
{biblatex}

\addbibresource{viitteet.bib}

\setlength{\parindent}{4ex}
\setlength{\parskip}{0ex}% Kappaleiden välin asetus

\linespread{1.3}

\addto\captionsrussian{% Replace "english" with the language you use
  \renewcommand{\contentsname}%
    {Оглавление}%
}

\addto\captionsrussian{% Replace "english" with the language you use
  \renewcommand{\bibname}
    {Список использованной литературы}
}

\addto\captionsrussian{% Replace "english" with the language you use
  \renewcommand{\refname}
    {Список использованной литературы}
}




\setlength{\voffset}{-0.02in}
\setlength{\hoffset}{0.57in}
\setlength{\marginparsep}{0pt}

% \usepackage{graphicx}
% tai \usepackage{graphics}
% kuvatiedostojen liittämiseksi

\begin{document}

\begin{titlepage}
\noindent
\begin{center}	
  \setlength{\parindent}{0mm}
  \sloppy
  \large \textsc{Хельсинкский университет}
  \vspace{5mm}

  \huge \textbf{Истина и ложь в нарративе романа \textit{Мелкий бес}}

%TEOSTEN NIMET KURSIVOIDAAN
%-artikkelit, novellit, runot, lainausmerkkeihin

  \vspace{2mm}
  \textcolor{blue}\hrule
    \vspace{2mm}
    \large  Istina i lož' v narrative romana \textit{Melkij mes}
    \linebreak \vfill   

  
%\vspace{20mm}
	
  \end{center}
  \vspace{15mm}
	\vfill
	
    \begin{flushright}
    	Tapio M. Pitkäranta \\
		Pro Gradu \\
		Venäjän kieli ja kirjallisuus \\
		Nykykielten laitos \\
		Helsingin yliopisto \\
		\today
	\end{flushright}
\end{titlepage}

%Valitaan tämän osuuden kieli
\selectlanguage{russian}

\tableofcontents
\thispagestyle{empty}

\pagebreak

%%% Oma teksti tämän jälkeen.
%%% Tyhjä rivi kappalten väliin.§] omiin tarkoituksiinsa.

%artikkeli, runo lainausmerkit
%erilliset teokset kursiivilla


\section{Введение}
% * <ida.reini@gmail.com> 2018-04-09T09:37:14.492Z:
% 
% tuli tuossa krisun työtä lukiessa mieleen, että olisi kiva nähdä ihmisten tutkimuskysymys vielä uudestaan, niin osaisi sanoa onko teksti relevanttia vai ei. eli jos on aikaa niin mun mieliksi voi lisätä vvedenien
% 
% ^.

% Данная работа посвящена изучению отношений между романа Федора Сологуба <<Мелкий бес>> и сологубской теории искусства, которой он развивается в разных статьях, например,  <<Не постыдно ли быть декадентом>> (1896), <<Демоны поэтов>> (1907) <<Театр одной воли>> (1908) и <<Искусство наших дней>> (1915). Особое внимание уделяется сологубским мифом о Дульцинее и Альфонсе, который основан на романе <<Дон Кихот>>. В этом мифе Сологуб соединяет быт с искусством, грех со святостью и декадентство с символизмом. Многие исследователи уже заметили параллели  романа Сологуба с <<Дон Кихотом>>, а в данной работе изучаем подробно как эта тематика влияет на нарративе романа.

\section{Федор Сологуб в контексте русского модернизма конца XIX -- начала XX веков}

% Творчество Федора Сологуба развивалось в рамках декадентства, русского модернистского литературного направления конца XIX века.
% Самый значительный из сборников Сологуба – «Пламенный круг».
% Главнимы мотивами в поэзии Сологуба являютя смерть,  

% Пессимизм, неверие в возможность изменить социальную жизнь
% беспомощность, малых возможностях человек
% индивидуалиста рубежа веков
%  не просто осознает, но и всячески культивирует свою отчужденность от общества.
% tässä näkyy tiukka modernismin vaikutus. ETSI JOKU LÄHDE
% «недотыкомка серая» (1899)
% неотвязное наваждение, рожденное суеверием, диким, косным бытом, житейской пошлостью и отчаянием.   Tästä sanoo Blok jotain??

% Частая тема поэзии Сологуба — власть дьявола над человеком («Когда я в бурном море плавал», 1902; «Чертовы качели», 1907). (431)
% дьявол у Сологуба, как и у Бодлера, символизирует не только зло, царящее в мире, но выражает и бунтарский протест против обывательского благополучия и успокоенности.

%  Поэтический словарь Сологуба пестрит словами смерть, труп, гроб, прах, склеп, могила, похороны, тьма, мгла.
%  «Ибо все и во всем — Я, и только Я, и нет ничего, и не было и не будет», — таким утверждением в духе солипсизма завершается предисловие поэта к этому сборнику. Это же предисловие открыло первый том «Собрания сочинений» Сологуба в издательстве «Сирин».


\section{Тематика романа}

В этой главе рассматриваем то, как другие исследователи осматривали тематику романа Мелкий бес. Мы обратимся особое внимание на тематику истины и лжи и реферируем теорию неомифологического романа З. Г. Минц. Дихотомия истина – ложь нам особенно интересен, так как она тесно связано с проблематикой нарратива и нарратора в романе.

\subsection{\emph{Мелкий бес} как \enquote{неомифологический} роман}

В. В. Ерофеев в своей статье \citetitle{jerofeev1985} (1985) анализирует роман \emph{Мелкий бес}. Его основным положением в статье является то, что \emph{Мелкий бес} Ф. Сологуба -- роман переходного периода в пути к модернистской прозе, который относиться пародийно к русской романной традиции, особенно к критическому реализму, но также к творчеству Гоголя, Пушкина и Достоевского. \parencite[145.]{jerofeev1985}


Ерофеев полагает, что \enquote{автор романа не приемлет той жизни, о которой повествует}. Повествователь в романе описывает идиотизм и садизм персонажей, пошлость жизни в маленьком городе и относится ко всему этому по большей части иронично. Образ автора, по мнению Ерофеева, образ прогрессивной фигуры, у которого прогрессизм -- \enquote{безнадежный, бесперспективный \enquote{прогрессизм}}. \parencite[146.]{jerofeev1985}


По Ерофееву, на основе традиционного русского романа является понятие о соответствии красоты, истины и добры, и несмотря на критическое отношение автора к социальной действительности, у автора все-таки есть надежда, что есть путь к лучшему обществу в будущем. Эта надежда в романе Ф. Сологуба превратилась в тоску, которая не может совершаться.  \parencite[158.]{jerofeev1985}



Чтение Ерофеева, хотя оно не настолько обширное, не противоречит чтения З. Г. Минц, которая в  своей статье \citetitle{mints2004} \parencite*{mints2004} рассматривает вопрос о романе Сологуба как о части традиции \enquote{неомифологических} текстов, традиции, которая достигнет кульминации в романе Андрея Белого \enquote{Петербург}. \enquote{Неомифологический} роман по Минц -- в тесном отношении к предыдущей традиции: \enquote{Ориентация на архаическое сознание непременно соединяется в \enquote{неомифологических} текстах с проблематикой и структурой социального романа, повести и т. д., а зачастую — и с полемикой с ними.} \parencite[60.]{mints2004} 


На основе мифопоэтического текста является понятие \enquote{панэстетизма}. В \enquote{панэстетизме} сущность мира -- Красота. И для символистов на рубеже XIX -- XX веков -- самая высшая красота -- искусство. Искусство также является единственным средством, через которое можно получить истинное знание о мире. Для символистов сам мир был мифом или символом и символистское искусство, творя мифы, при этом превращает мир. На картину мира символистов также влияла философия Владимира Соловьева и его идея о триадной сущности мира. По Соловьеву, мир находится в постоянном состоянии битвы, где Божественный Космос борется с Хаосом за Душу мира. Из этого следует миф о мире, который находится в состоянии изменения и в котором бесконечно следуют друг за другом \enquote{теза}(совершенное состояние, где существует только Божество), \enquote{антитеза}(мир разбивается Хаосом) и  \enquote{синтеза} (небесное и земное соединяются опять в всеединое). Из этого мифа о мире и следует диалектический характер \enquote{неомифологического} текста, где на вид противопоставляющие понятия существуют в одном тексте: \enquote{\enquote{Земное} хотя и ощущается как \enquote{отблеск} и \enquote{отзвук} небесного, как \enquote{антитеза} духовной \enquote{тезы} бытия, однако наделяется самостоятельностью, самоценностью и не меньшей, чем \enquote{теза}, ролью в становлении мирового универсума.} \parencite[71.]{mints2004}


\enquote{Неомифологический} текст принимает осознанно материал из разных источников и традиции и использует их чтобы построить свою картину мира, свои миф. Поэтому \enquote{неомифологический} текст -- в сущности интертекстуальный или полигенетичный. В этом дуализм символистов различается от дуализма романтики, в котором противопоставление между предметного мира и мира искусства является несовместимым. У символистов, Сологуба в том числе, предметный, нынешний мир мифологизируется и вместе с другими текстами-мифами формируется в текст-миф художественного текста. Как показывает Минц, одним из типично важных источников \enquote{неомифологического} мифа является современная общественная действительность. Для Ф. Сологуба, это материал частично из его собственной жизни. Минц определяет, что в романе \emph{Мелкий бес}, миф Передонова формируется из следующих источников:

\begin{enumerate}[itemsep=0mm, label=\asbuk*)]
\item Текст-миф реализма о жизни в периферии (Гоголь, \enquote{Человек в футляре} Чехова).
\item \enquote{Мертвые души}.
\item Безумие, хаос \enquote{Бесы}, \enquote{Идиот} Достоевского.
\item \enquote{Пиковая дама} Пушкина, безумие, мотивы игровых карт.
\item Его собственное стихотворение \enquote{Недотыкомка}, безумие обычной мещанской жизни
\item Мифы, которые Передонов творит сам, мания преследования.
\end{enumerate}

Другие исследователи добавляли в списке также, в частности, \emph{Дон Кихот} Сервантеса \parencite{bagno2009} и разные библейские мотивы и мифы \parencite{kobrinski2013}, \parencite[272]{silard1984}.


 
Как показывала Минц, одним из текст-мифов действительно является традиция реалистического русского романа, которую Ерофеев описывает в своей статье, но в романе \emph{Мелкий бес} он просто частью более обширного мифа высшего уровня, вместе с другими, в описываемой Минц системе \enquote{неомифологического} текста.


По мнению Ерофеева, Минц \enquote{излишне мифологизирует} роман, и он полагает, что мифологический материал в сюжетной линии с Людмилой и Сашей не связан напрямую с древней мифологией, а с современной Сологубе французской и русской литературной модой \parencite[152]{jerofeev1985}.

М.М. Павлова довольно убедительно показала, что на раннее творчество Сологуба действительно влиял европейский натурализм и декаданс конца XIX века, особенно роман Ж. К. Гюсманса \emph{Наоборот} (\emph{À rebours}). Павлова пишет: \enquote{Он внес также существенные изменения в первоначальный замысел \emph{}{Мелкого беса}, дополнив его \enquote{эротической} сюжетной линией: \enquote{парфюмерным} романом Людмилы и Саши.} \parencite*[168.]{pavlova2007} При этом Сологуб в своей неопубликованной статье 1896 года \citetitle{ref:sologub2007} обширно обсуждает французскую литературу.

Ерофеев, однако, не учитывает, что в системе \enquote{неомифологического} романа, который описывает Минц, текст-миф современной французской литературы может подобным образом стать частью мифологии и «картины мира» \enquote{неомифологического} текста. Таким образом, наблюдение Ерофеева о влиянии литературной моды не опровергает гипотезу Минц, а даже укрепляет ее, так как он добавит в систему еще один текст-миф, который вводит еще одну свойственную \enquote{неомифологическому} роману дихотомию между мифами.

Мы соглашаемся с Минц, что \emph{Мелкий бес} -- \enquote{неомифологический} роман и, следовательно, не можем согласиться с Ерофеевым с тем, что роман просто пародия, так как это означал бы, что в романе существует какая-то единая авторская позиция или точка зрения. Как пишет Минц:

\begin{quote}
Поэтому именно отнесенность смысла символистского текста к универсальному «мифу о мире» оказывается единственной относительно авторитетной точкой зрения этого текста, наиболее слышным «голосом» в «полифоническом» произведении. \emph{Голос автора} в символистском «неомифологическом» произведении \emph{есть определение места изображаемого в универсальном космогоническом мифе.}

\parencite[77.]{mints2004}
\end{quote}

Вместо того, противопоставление между новым романом и традицией является лишь одной из нескольких дихотомии, тезой и антитезой, на основе которых в текст-мифе стремятся к синтезу в картине мира романа.

В роли такой же дихотомии может быть и декадентство и символизм. В 1890-е годы в России разгорелась полемика о формах нового искусства. Зинаида Гиппиус писала в статье, опубликованной в журнале \emph{Северный Вестник} 1896 г., так: 

\begin{quote}
Символизм, прежде всего, диаметрально противоположен декадентству. Быть может, даже не стоило бы упоминать о декадентстве рядом с символизмом. [---] [Э]ти два понятия так печально смешались в умах людей даже наиболее почтенных, что невольно хочется разделить их навсегда. [---] Декадентство боялось смерти и умерло, больное и слабое. Декадентство боялось разума, чистоты понимания. Символизм весь в свете разума, в его широком и ясном спокойствии. \footnote{Северный вестник. № 12. 1896. с. 235-246. цит. по \cite[150]{pavlova2007}.}

% * <ida.reini@gmail.com> 2018-04-10T07:28:07.124Z:
% 
% miksi tämä on viitattu footnotena?
% 
% ^.
\end{quote}

По мнению З. Г. Минц, в исторической перспективе, символизм и декадентство не противостояли так диаметрально, а на практике они были лишь разными \enquote{полюсами притяжения}, между которыми писатели двигались, принимая разные характеристики того или другого полюса \parencite[62.]{mints2004} В романе Ф. Сологуба можно видеть попытку синтеза этих полюсов. Об этой попытке он написал в своей неопубликованной статье, таким образом:

\begin{quote}
Уже и теперь мы видим литературные произведения, в которых символизм облекался в формы декадентские, и которые производят глубокое впечатление. Будущее же в литературе принадлежит тому гению, который не убоится уничижительной клички декадента и с побеждающей художественной силой сочетает символическое мировоззрение с декадентскими формами.

\parencite[501.]{ref:sologub2007}
\end{quote}

П. П. Перцов, критик, поэт и публицист писал в декабре 1897 года в письме В. В. Розанову о своих мыслях о новых направлениях в литературе и о декадентстве. Он тоже заметил новое направление и видел в Сологубе глашатай этого направления: 


\begin{quote}
Что Соллогуб [sic!] вообще больной человек – это ясно, но этого я ему не ставлю в вину: напротив – именно в этой-то слабости и сила его. [---] Он, Соллогуб дает свое имя новому периоду русской поэзии – последнему, \enquote{упадочному}, слабому и умирающему, конечно.

%katso alkup. lähde????

\parencite[536.]{pavlova2016}
\end{quote}

Перцов видел прямую линию в поэзии из Пушкина через Фета до Сологуба, и он считал их самыми типичным представителями своих периодов, однако, соглашая с Сологуб, он тоже считал,сим что в прозе такого пока не была. Перцов таким образом предвидел следующий шаг Сологуба в литературе.

%Перцов keskustelee dekadenssin voitosta ja sen tulevaisuudesta, suhteessa vanhaan kirjallisuuteen, romantiikkaan ja realismiin. \parencite[534]{pavlova2016}



\subsection{Дихотомия истинa – лoжь}


% И все те же и те же иллюзии повторялись и мучили его. Варвара, 
% тешась над Передоновым, иногда подкрадывалась к дверям той горницы,
% где сидел Передонов, и оттуда говорила чужими голосами. Он 
% ужасался, подходил тихонько, чтобы поймать врага, — и находил Варвару.
% — С кем ты тут шушукалась? — тоскливо спрашивал он.
% Варвара ухмылялась и отвечала:
% — Да тебе, Ардальон Борисыч, кажется.
% — Не все же кажется, — тоскливо бормотал Передонов, — есть
% же и правда на свете.
% Да, ведь и Передонов стремился к истине, по общему закону всякой
% сознательной жизни, и это стремление томило его. Он и сам не сознавал,
% что тоже, как и все люди, стремится к истине, и потому смутно было его
% беспокойство. Он не мог найти для себя истины, и запутался, и погибал. (200)

\section{Нарратив в романе \textit{Мелкий бес}}

В этом главе рассматриваем структуру нарратива в романе Федора Сологуба \emph{Мелкий бес}. Некоторые исследователи характеризовали сюжет романа как антисюжет -- очень упрощенная история в котором почти ничего не происходит: 
\enquote{Внешнего действия в \emph{Мелком бесе} мало. Постепенно сходит с ума его главный герой, учитель гимназии Передонов, умственно ограниченный, угрюмый и недоброжелательный человек.}
\autocite[432.]{grigorjev1983} С другой стороны, Стивен Хатчинс, описывал роман, наоборот, термином \enquote{narrative excess}, избыток нарратива. По его мнению, в романе можно читать множество \enquote{лишних} сюжетов, таких, как например слухов, ложей и рассказов, которые придумают персонажи. \parencite[111--114.]{hutchings1997} Мы согласны с Хатчинсом в том, что структура романа сложнее чем сначала кажется. Текст-миф романа – многоуровневый и построена полигенетично из многочисленных текстов-мифов, которые сложно соотнесены друг со другом. 


Как мы заметили выше, В. В. Ерофеев видел \emph{Мелкий бес} как пародия реалистического романа, но отметил, что образ автора в романе неясен и ведет к осушению хаоса, может быть неосознанно автором. Хотя мы не можем полностью согласиться со чтением Ерофеева, его замечание о том, как повествователь в романе – многолик, очень актуален. Мы полагаем, что внутренняя дихотомия, свойственная \enquote{неомифологическому} роману, не может не влиять на структуру и нарратив данного романа и поэтому данная глава посвящена этому проблему.


\subsection{Терминология}

В нашем анализе, мы употребляем терминологию из монографии Вольфа Шмида \emph{Нарратология} (2008), так как данная книга является самим ясным и систематичным введением по этой теме на русском языке, найдено нами. Потому что терминология, употребляемая Шмидом, немного отличается от традиции русского литературоведения, в частности от терминологии Женетта, мы описываем значению некоторых терминов.
  
По мнение Шмида, термин \enquote{повествователь} -- проблематичен, так как он
определяется по-разному. При этом, и в советском и в современном
литературоведении термины \enquote{образ автора} и \enquote{образ повествователя} употребляли как
синонимы \parencite[67--68.]{schmid2008}  Как мы заметили выше, у В. В.
Ерофеева именно этот термин смешивается с термином «автор». Мы в нашем анализе
употребляем термин нарратор, потому-что он является как можно больше
нейтральным. Особенно в \enquote{неомифологическом} романе, как мы показывали выше, мы не можем определить какой-то единой авторской позиции.  Когда речь идет о разных уровнях нарратива, употребляем термины
первичный нарратор, вторичный нарратор и т.д. Первичным называется тот нарратор,
который  обрамляется истории и вторичным повествователя интекста или вставной истории.

В категоризации Шмида противопоставление \enquote{диегетический -- недиегетический} также немного отличается от терминологии Женетта. По словам Шмида: \enquote{... если \enquote{я} (нарратора – \textit{TMP}) относится только к акту повествования, то нарратор является недиегетическим, если же \enquote{я} относится то к акту повествования, то к повествуемому миру — диегетическим.} \parencite[84]{schmid2008}.

% \subsubsection{Fokalisaatio}
% Kertojan diskurssi, henkilön diskurssi ja kertojan-henkilön diskurssi

\subsection{Образ нарратора}

%Miksi haluamme analysoida kertojaa? 

Образ нарратора интересует нас потому что он важен, когда анализируем дихотомию правды и обманы в романе. В реалистическом романе нарратор традиционна был объективным, недиегетическим, всезнающим и рассказывал правду. Возможным был и личный, диегетический нарратор, но в таком случае объективность нарратора снижается, и читатель может сомневаться в словах нарратора. 


Что мы знаем о нарраторе в романе \emph{Мелкий бес}?
Есть ли нарратор так \enquote{многолик} как утверждает Ерофеев? В нарративе романа находятся и эксплицитные и имплицитные признаки изображения нарратора. Нарратор кажется нам большей частью всезнающим, и он может связать информацию о чувствах, мыслях и мотивациях персонажей, как например в этом отрывке, где нарратор описывает мысли Передонова:  \enquote{Он не знал, о чем говорить с Мартою. Она была ему нелюбопытна,
как все предметы, с которыми не были кем-то установлены для него
приятные или неприятные отношения.
} \parencite[18]{sologub2004}. Однако, иногда знания нарратора являются более условными: \enquote{Хоть он и узнал наверное, что смеялись не над ним, но в нем осталась досада, — так после прикосновения жгучей крапивы долго остается и  возрастает боль, хотя уже крапива и далече.} \parencite[19]{sologub2004}. Употребление слова \enquote{наверное} подчеркивает условность и неуверенность информации. С другой стороны, данная цитата является амбивалентной – читатель не может быть уверен, если это текст персонажа или текст нарратора, который сказан с точки зрения персонажа. Однако, описание растениях является в нарративе повторяющим мотивом, и это делает только нарратор. В цитате, эмоция Передонова сравнивается с прикосновением жгучей крапивы и поэтому можно полагать, что это текст именно нарратора, а не Передонова.

Нарратор обычно использует преимущественно нейтральный регистр литературного языка. Язык нарратора является в оппозиции с персонажами, которые часто употребляются в речи диалектные слова, каламбуров, поговорок и, в случае Передонова, даже чары. Согласно Джулиан Коннолли, в протяжении сюжета, речь персонажей постепенно лишается когерентность, и это совпадает тематически с повышением хаоса \parencite[358--359]{connolly1981}.


Иногда, особенно высказывая свое мнение о персонажах или событиях, нарратор переходит к высшему регистру языка: \enquote{– и воистину в нашем веке надлежит красоте быть попранной и поруганной.} \parencite[51]{sologub2004}.

%----i
% * <ida.reini@gmail.com> 2018-04-10T07:50:23.076Z:
% 
% ?
% 
% ^.

Однако нарратор знает также язык местных жителей города. Доказательство этого мы видим впервые в главе IV, где нарратор упомянет вскользь, что живет в одном городе с персонажами: \enquote{ – А вы ее зовите Клавдюшкой.
Варваре это понравилось. Она повторяла:
— Клавдюшка, дюшка.
И смеялась скрипучим смехом. Надо заметить, что дюшками в 
нашем городе называют свиней.} \parencite[33.]{sologub2004} Эта показательная фраза повторяется в той же главе еще раз: \enquote{Поэтому она чаще ездила на извозчиках,
хотя больших расстояний в нашем городе не было.} \parencite[34]{sologub2004}. Эти заметки показывают особые черты, индивидуальную личность,  нарратора. Он живет в \enquote{нашем городе} и знает наречие этого города. Одновременно с повышением личности, снижается объективность нарратора.

Отношение нарратора к персонажам обычно является пренебрежительным. Он недооценивает и осуждает персонажей эксплицитно, как например в сцене, где нарратор описывает телесную любовь Варвары и Передонова: 
\enquote{И это восхитительное тело для этих двух пьяных и грязных людишек являлось только источником низкого соблазна.} \parencite[51]{sologub2004}.
Имплицитно нарратор осуждает персонажей в частности таким образом, что в нарративе отрицательные черты и характеристики персонажей, идиотизм, грубое внешность, садизм, подчеркиваются. В этом тоже видно личность нарратора, так как у него есть мысли и мнения о персонажах и событиях романа. Однако, иногда и в тексте нарратора видно влияние персонажей и их садистические черты передается тоже на нарратора. Приведем как пример описание внешности Грушины, когда ее впервые знакомится с читателем:


\begin{quote}
На беглый взгляд она не то
чтоб казалась очень грязною, а производила такое впечатление, словно
она никогда не моется, а только выколачивается вместе со своими 
платьями. Думалось, что если ударить по ней несколько раз камышовкою,
то поднимется до самого неба пыльный столб.

\parencite[33.]{sologub2004}
\end{quote}

В цитате, приведенной выше, нарратор не просто описывает что Грушина казалась очень грязной, а  связывает избиение прутом персонажа с символизмом пыля с помощью гиперболы: ударяя ее, «поднимется до самого неба пыльный столб». Здесь мы обнаруживаем тематическое совпадение нарратора с образом Людмилы, Саши Пильникова и Передонова, так как Передонов любит хлестать своих учеников и в отношении Саши и Людмилы тоже видно явная садомазохистская тенденция:
\begin{quote}
И уже смотрел ясно и спокойно, как Людмила опускала его рубашку, обнажала его плечи, ласкала и хлопала по спине.

\parencite[208]{sologub2004}.

Хотелось что-то сделать ей, милое или больное, нежное или
стыдное, — но что? Целовать ее ноги? Или бить ее, долго, сильно, длинными гибкими ветвями?

\parencite[211]{sologub2004}.
\end{quote}

Согласно Т. Венцелова, у русских символистов, начиная с стихотворение В. Брюсова \enquote{Демоны пыли} (1899)\footnote{ Согласно Иоанне Брюсове, поэт получил вдохновение следя как жена стирала пыль в комнате и он видел как пылинки плясали в солнечном луче. И здесь бытовая пыль преобразилась в символическую. \parencite[600]{brjusov1973}.}, пыл ассоциируется с демонами, хаосом, энтропией и сама Вершина в романе связывается с бесом женского пола \parencite[42--44]{venclova2012}. 

Согласно Н. Г. Пустыгине в статье \enquote{Символика огня в романе Федора Сологуба \emph{Мелкий бес}}, символы пожара — дыма — пыли — золы — праха являются в романе семантически взаимозависимыми и взаимозаменяемыми и поэтому образ Вершины является в романе двойствен:

\begin{quote}
Прежде всего в нем (персонаже Вершины – \textit{TMP}) заключена символика смерти, а также и всей действительности (причем дым в последнем случае соотносится с «покрывалом майи»). Символом дыма как бы еще раз подчеркивается иллюзорность, «кажимость» вещного мира, одной из его многочисленных ложных оболочек. Одно «обличие мира», сгорев, оставляет после себя прах; круг, таким образом, замыкается.

\parencite[134.]{pustygina1989}

\end{quote}

С выражением «покрывало майи» имеется в виду понятие из философии Артура Шопенгауэра, которое он взял из ведических текстов индуизма. Майя в Шопенгауэрской философии — покрывало обмана, которое скрывается иллюзорность существования. Таким образом, символике пыла и дыма и образ Вершины связывается с дуализмом Сологубовского мировоззрения, в котором  предметный мир – мир иллюзии и лжи, а истинный мир можно видеть только через творение художника.

Опять мы сталкиваемся с проблемой образа нарратора. Мы полагаем, что диалектичность образа нарратора именно отражает мифа о мире -\enquote{неомифологического} романа и является неизбежным следствием того. Нарратор не может быть никаким другим чем двусмысленным, двухполюсным. С одной стороны, в нарраторе отражается натуралистический и критический реалистический роман. Он описывает пошлость маленького города, провал человечества, но он также является сам жителем этого города. С другой стороны, в нарраторе есть и поэт-символист, который ведет текст к символистическому, мифологическому уровню повествуя об обычном быте. В образе нарратора совмещается, как это выразил Сологуб: \enquote{символическое мировоззрение с декадентскими формами.}



% Myös ympäristön kuvauksissa esiintyy sama:
% Странная, жестокая улыбка сквозь слезы озарила ее лицо, как 
% ярко-пылающий луч на закате сквозь последнее падение усталого дождя.(142) valo pölyn läpi, 
% тряпкою, лентою, веткою, флагом, тучкою, собачкою, столбом пыли на улице, и везде ползет и бежит за Передоновым, — измаяла, истомила его зыбкою своею пляскою. Хоть бы кто-нибудь избавил, словом каким (198)




% \subsubsection{Недиегетический нарратор}

% Kysymys

% \subsubsection{Картина мира нарратора}

% \subsection{Ympäristön kuvaus, henkilöt vs kertoja}

\subsection{Ирония нарратора}

% \enquote{В повестях и романах все глупости пишут.} \parencite[54]{sologub2004}


Теперь собираемся показать, как нарратор романа относится иронически к персонажам и обратимся внимание на противопоставление между ироническом нарратором и фокализируемыми нарратором персонажами. В нарративе, ирония связана с фокализацией. Согласно Шмид, \enquote{ирония основывается не просто на внешней точке зрения, а на одновременности двух оценочных позиций, на симультанной \enquote{внутринаходимости} и \enquote{вненаходимости}, пользуясь понятиями Бахтина} \parencite[118]{schmid2008}.  Хотя иронический нарратор не было необычным приемом уже во время романтизма и появляется нередко также у Достоевского, у Сологуба ирония является очень важным приемом, который тесно связано с его мировоззрением и теорией искусства.

Анализируем два показательных отрывка где видно фокализация, ирония и противопоставление между нарратором и персонажами.

\begin{quote}
Дома ждала Передонова важная новость. Еще в передней можно
было догадаться, что случилось необычное, – в горницах слышалась 
возня, испуганные восклицания. Передонов подумал, – не все
готово к обеду: увидели – он идет, испугались, торопятся. Ему стало
приятно, – как его боятся! Но оказалось, что произошло другое. 
Варвара выбежала в прихожую и закричала:

– Кота вернули!

\parencite[171.]{sologub2004}

\end{quote}

В этом отрывке ироническое отношение нарратора показано с помощью фокализации. Первое предложение отрывка написано с точки зрения нарратора – с нарраториальной точки зрения. Точка зрения не может быть персональным, так как Передонов же не видит в будущее и не знает заранее, что его дома ждет новость. Потом фокализация перейдет на персональный и нарратор описывает ситуацию сначала с помощью информации слышно ушами Передонова – \enquote{слышалась 
возня, испуганные восклицания}. Доказательством передоновской фокализации здесь является кроме сенсорных восприятий Передонова также внутренняя психологическая точка зрения. Передонов сделает вывод, что шум оттого, что его боятся. Потом возвращает более объективная нарраториальная точка зрения: \enquote{Но оказалось, что произошло другое.} Через прямую речь Варвары наконец открывается, что была эта \enquote{важная} новость.
Глядя на ситуацию объективно, возвращение кота вряд ли важная новость. Однако, в нарративе вход Передонова описывается драматично как сцена из повести Эдгара Аллана По – страх, крики, торопящиеся люди. Иронический тон нарратора показывается  в оппозиции между нарроториальной и персональной фокализациями, или \enquote{на одновременности двух оценочных позиций} и через гиперболизации нарратора – важная новость не оказывал так важной. 

Во втором цитате персонажами являются Людмила и директор гимназии Хрипач. В сцене, Хрипач пытается узнать кто говорит правду об отношении Саши, Людмила или Передонов. 

\begin{quote}
Главное ее побуждение было,
конечно, сочувствие к мальчику, оскорбленному таким грубым 
подозрением, — желание заменить Саше отсутствующую семью, — и, 
наконец, он и сам такой славный, веселый и простодушный мальчик. 
Людмила даже заплакала, и быстрые маленькие слезинки 
удивительно-красиво покатились по ее розовым щекам, на ее смущенно-улыбающиеся
губы. 

\parencite[239.]{sologub2004}
\end{quote}

Здесь обнаружим как нарратор описывает рассказ Людмилы с внешней позиции с иронией. Ирония видно в употреблении частиц, как например \enquote{конечно} и в выражении \enquote{слезинки удивительно-красиво покатились}. Нарратор выразит свое циничное недоверие к ложному рассказу Людмилы, но противоположно тому, фокализация перейдет к директору гимназии, начиная с его чувством обоняния:

\begin{quote}

— и нежным ароматом повеяло на Хрипача. И Хрипачу
вдруг захотелось сказать, что она «прелестна, как ангел небесный», и
что весь этот прискорбный инцидент «не стоит одного мгновения ее 
печали дорогой». Но он воздержался.

И журчал, и журчал нежный, быстрый Людмилочкин лепет, и 
развеивал дымом химерическое здание передоновской лжи. Только 
сравнить — безумный, грубый, грязный Передонов — и веселая, светлая,
нарядная, благоуханная Людмилочка. Говорит ли совершенную 
Людмила правду или привирает — это Хрипачу было все равно, — но он 
чувствовал, что не поверить Людмилочке, заспорить с нею, допустить 
какие-нибудь последствия, хоть бы взыскания с Пыльникова, — значило
бы попасться впросак и осрамиться на весь учебный округ. Тем более
что это связано с делом Передонова, которого, конечно, признают 
ненормальным.

\parencite[240.]{sologub2004}
\end{quote}

Опять знаками внутренней, персональной фокализации являются ощущения и идеология персонажа. Вторым знаком является употребление уменьшительно-ласкательной формы имени Людмилы.  При этом, Хрипач сравнивает, свойственным для \enquote{неомифологического} романа средством интертекстуального цитата, Людмилы с персонажами двух отдельных произведении Лермонтова, поэма \enquote{Демон} и стихотворения \emph{Тамара}. Цитаты в тексте выделены кавычками. Однако Хрипач неточно цитирует их, и не помнит, как продолжает цитата из стихотворения \emph{Тамара}: \enquote{Прекрасна, как ангел небесный, | Как демон, коварна и зла.} \parencite[194]{lermontov2000}. С другой стороны, Хрипач указывает на Передонова фразой \enquote{химерическое здание передоновской лжи}. Химера – в греческом мифологии трехголовoe, огнедышащее чудовище. У нее головы льва и хвост дракона. Согласно А. Ханзен-Леве, в поэзии декадентов, Химера символизирует часто ложь и измена, но именно в форме женщины \parencite[200]{hansen-love1999}. Накопление и совмещение разных текстов-мифов в фокализации Хрипача вместе с противоречием между ироническим тоном нарратора и мыслями персонажа произведут эффект полифонии, который, как показано выше, свойствен \enquote{неомифологическому} роману.

Второе противопоставление в цитате является между Людмилой и Передоновом, то есть, между истиной и ложью. С точки зрения нарратора ясно, что Людмила врут, а с точки зрения Хрипача, прекрасная Людмила, образ которой у Хрипача построена на основе поэзии Лермонтова и более широко на основе женщины романтической литературы, не может быть ложной и обманчивой, тем более, когда ее сравнивают с грубым, грязным, «ненормальным» Передоновым. Нарратор описывает обычную, бытовую жизнь маленького города, пошлость. Но когда фокализация перейдет к Хрипачу, пошлость преображается в миф путем интертекстуальные ссылки. Цитаты Хрипача здесь имеются, по категоризации З. Г. Минц, метонимичное отношение с цитируемыми произведениями \parencite[396]{mints1973}. Цитата – знак целого стихотворении Лермонотова, может быть даже знак творчества Лермонтова, и она совмещает в образ Людмилы не лишь содержание цитированного строка, а содержание стихотворении. Тамара Лермонтова в стихотворении – не просто красивая, но также коварная, и она заманивает мужчин, только чтобы потом уничтожить их. Вторая, дополнительная, строка лермонтовской цитаты в сущности сближает точки зрения Хрипача и нарратора – приведет тезу и антитезу к синтезу.  


В цитате противопоставлены иронический и лирический образ мира и искусства. Хрипач – представитель лирического образа. Он соединяет красоту Людмилы с правдой и грубость Передонова с ложью. Это же классическое греческое понятие \enquote{калокагатия} (\foreignlanguage{greek}{καλοκαγαθία}), сочетание физической красоты и нравственного добра.  Нарратор – представитель иронического образа мира, так как он открывает в нарративе ошибочную логику Хрипача. Об этом противопоставлении писал Федор Сологуб в своей статье \textit{Старый чорт Савелич}, опубликованным в 1907 году в журнале \emph{Перевал}:  


\begin{quote}
Всякая поэзия хочет быть лирикою, хочет сказать здешнему,
случайному миру \textit{нет}, и из элементов познаваемого выстроить
мир иной, со святынями, «которых нет».
[---]
Но всякая истинная поэзия кончает ирониею.

\parencite[164.]{sologub1991}
\end{quote}

По мнению Ф. Сологуба, только ирония способен выявить истинное описание этого враждебного, ложного мира, видеть за \enquote{покрывалом Майи}. Но лирика тоже важна, так как с помощью лирики, поэт творит свой собственный мир:

\begin{quote}
 Лирический поэт, говоря нет данному миру, говорит это для того, чтобы восхвалить мир, которого нет, который долженствует быть, которого Я хочу, который Я творю. Творю подвигом всей моей жизни. 
 
\parencite[170.]{sologub1991}
 \end{quote}
 
 Здесь, как и в многих других произведениях этого периода, например, в статье \enquote{Я. Книга совершенного самоутверждения} (1906),  Сологуб подчеркивает именно личность поэта или художника единственным, который умеет творить новый мир и который может разорвать цепи судьбы и быть самостоятельным субъектом.


Хотя на основе \enquote{неомифологического} текста является \enquote{панэстетизим}, согласно которого Красота является сущностью мира, земная красота, представителем которой в цитате является Людмила, оказывается также ложной. А истинная красота можно твориться только путем искусства. Здесь является цель иронии в романе. Только путем иронии, может нарратор обнаруживать принципиальную противоречивость этого мира, который он описывает и невозможность найти истинную правду в мире быта и пошлости. Ироничное повествование смешивает категории добра и красоты, уродства и лжи и показывает, что персонажи романа нельзя классифицировать односторонно в добрых и злых, а что они тоже в сущности являются полигенетичными.  

% Ainut todellinen kauneus voi syntyä vain taiteen kautta. Tässä piilee ironisen kertojan tarkoitus romaanissa. Kertojan ironia paljastaa kuvaamansa maailman pohjimmaisen ristiriitaisuuden ja mahdottomuuden löytää todellista totuutta arjen, poshlostin ja maailmassa. Ironinen kerronta sekoittaa hyvyyden ja kauneuden, rumuuden ja valheen kategoriat ja osoittaa, että romaanin henkilöitä ei voi selittää yksipuolisesti tai kategorisoida yksinkertaisesti hyviksi tai pahoiksi, valheellisiksi tai totuudellisiksi, vaan nekin ovat luonteeltaan polygeneettisiä гетерогенность образов и сюжета.




% Katkelmassa sekoittuvat hyvän ja kauniin vastaavuus totuuden kanssa, antiikin kreikan kalogathia, joka oli myös panestetismin pohjalla

% Romaanissa kertojalla on tärkeä osuus tarinan totuuden arvioinnissa. 

% Tässä tulee ehkä myös metatason viittaus koko romaanin sisältöön. Voiko hyvän ja kauniin sekä totuuden välille vetää suoraan yhtäläisyysmerkin? Onko antiikin \enquote{калокагатия} (\foreignlanguage{greek}{καλοκαγαθία}) riittävän kattava periaate romaanin tulkinnassa?

% Ironia ja Puškin:

%  Здесь мы видим лирическое отношение к предмету, такого отношения не вызывающему. И в этом было обольщение для поэта, — обольщение лживое и опасное. 
 


% \begin{quote}
% Но всякая истинная поэзия кончает ирониею. Пламя лирического восторга сожигает обольстительные обличия мира, и тогда перед тем, кто способен видеть, — а слепые не творят, — обнажается роковая противоречивость и двусмысленность мира.
% \end{quote}

% \begin{quote}
%  Лирический поэт, говоря нет данному миру, говорит это для того, чтобы восхвалить мир, которого нет, который долженствует быть, которого Я хочу, который Я творю. Творю подвигом всей моей жизни. 
%  \end{quote}

\subsection{Структура пространства в тексте нарратора и персонажей}

\subsection{Структура романа?}

\subsection{Человек в футляре?}

\section{Заключение}

\nocite{*}


\printbibheading[heading=bibliography,title={Список использованной литературы}]
\printbibliography[heading=subbibliography,keyword={primary},title={Источники}]
\printbibliography[heading=subbibliography,notkeyword={primary},title={Исследовательская литература}]


\end{document}
